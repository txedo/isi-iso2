\section{Conclusiones}

Symbian fue de los primeros sistemas operativos que se implant� en dispositivos m�viles de bajo precio. Este hecho, junto a su capacidad de procesamiento en tiempo real y multi-tarea, hizo que consumidores particulares y desarrolladores se decantasen por esta plataforma. Su temprana aparici�n, casi sin competencia, y el apoyo de Nokia hizo que ambos ganasen una ventaja importante en el mercado (ver Tabla \ref{table:comparisonOS}). Sin duda Symbian marc� un inicio en la era de los \textit{smartphones} "`al alcance de todos"'.

No obstante, Nokia no supo aprovechar la ventaja que ya ten�a en el mercado innovando e investigando para motivar la evoluci�n de Symbian. Ello ha permitido que empresas como Apple y Google hayan recortado distancias: Apple apostando fuerte por la experiencia de usuario y Google por el desarrollo de aplicaciones. Esta fuerte competencia ha forzado a Symbian a liberar el c�digo de su sistema operativo para volver a ganar de nuevo el apoyo de la comunidad y evitar salirse del mercado.
