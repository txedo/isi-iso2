\section{Desarrollo en Symbian}

\subsection{Versiones de Symbian}

En los siguientes apartados se van a comentar las diferentes versiones que existen del sistema operativo Symbian, destacando sus caracter�sticas m�s importantes.

\paragraph{Symbian OS 6.0 y 6.1}

La versi�n 6.0 es la primera versi�n comercial de este sistema operativo, utilizada en el tel�fono m�vil \textit{Nokia 9120 Communicator} en el a�o 2001. A partir de esta versi�n, el sistema operativo de Psion pasa a llamarse Symbian y se encamina al desarrollo para tel�fonos inteligentes \textit{smartphones}. 

En esta versi�n se incluye el soporte para conectividad Bluetooth y el soporte para la ejecuci�n del conjunto de instrucciones de tipo ARM (\textit{Advanced RISC Machine}). Adem�s, se intenta el desarrollo de interfaces de usuario separadas para los \textit{smartphones} y para los dispositivos m�s convencionales, obteniendo dos referentes: la interfaz \textit{Quartz} y la \textit{Crystal}. Sin embargo, aunque se hab�a intentado un desarrollo com�n, se produjo una clara divisi�n, pasando Nokia a utilizar la primera de ellas y Sony Ericsson la segunda. \\
\indent Con esta separaci�n, se comienza a hablar de la serie de dispositivos \textbf{Series 80 1st Edition}, que son los pertenecientes a Nokia, y la serie \textit{UIQ 1.0 y UIQ 1.1}, pertenecientes a Sony Ericsson y Motorola, diferenci�ndose en el tipo de interfaz de usuario, aunque la plataforma y versi�n de Symbian utilizada es la misma en ambos.

En cuanto a la versi�n 6.1, se incorpora el soporte para una c�mara VGA con resoluci�n de 0.3 Mpx, utiliz�ndose por primera vez en el Nokia 7650, perteneciente a la serie \textbf{Serie 601st Edition}.

\paragraph{Symbian OS 7.0}

Esta versi�n se publica en el a�o 2003, a�adiendo soporte para IPv6, para la tecnolog�a EDGE (\textit{Enhanced Data Rates for GSM}) y para Java, siguiendo el est�ndar Java ME. Esta versi�n se utiliza con cualquier tipo de interfaz de usuario, incluyendo las series \textbf{Serie 80 2nd Edition}, \textbf{Serie 90 1st y 2nd Edition}, \textbf{Serie 60 2n Edition}, \textbf{UIQ 2.0} y \textbf{UIQ 2.1}.

Por otra parte, �sta es la primera versi�n de Symbian con un problema de seguridad grave, ya que es infectada por el gusano llamado \textit{Cabir}, capaz de transmitirse a otros m�viles con Symbian OS a trav�s del Bluetooth.

\paragraph{Symbian OS 8.0}

La versi�n 8.0, lanzada en el a�o 2004, cuenta con dos kernel del sistema operativos diferentes, el EKA1 y el EKA2, aunque �ste ultimo no se incorpora hasta la versi�n 8.1b. En esta versi�n se mejoran las capacidades multimedia del sistema operativo, teniendo mas rendimiento en aplicaciones que utilizan \textit{streaming} e incorporando OpenGL con gr�ficos vectoriales; se permite la administraci�n remota del dispositivo; se soporta CDMA (Code Division Multiple Access) y la tecnolog�a 3G y, por �ltimo, se aumentan los perif�ricos de entrada y salida, dando soporte a sintonizadores digitales de TV (DVB-H), reconocimiento de huellas, etc.

Esta versi�n se incluye en las serie \textbf{Series 60 2nd Edition 2.6}.

\paragraph{Symbian OS 8.1}

\paragraph{Symbian OS 9.0}

\paragraph{Symbian OS 9.0}

\paragraph{Symbian OS 9.1}

\paragraph{Symbian OS 9.2}

\paragraph{Symbian OS 9.3}

\paragraph{Symbian OS 9.4}

\paragraph{Symbian OS 9.5}

\subsection{Entornos de desarrollo, API y SDK de Symbian}


\subsection{Firma y publicaci�n de aplicaciones}