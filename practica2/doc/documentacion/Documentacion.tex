% Tipo de documento. En este caso es un art�culo, para folios A4, tama�o de la fuente 11pt y con p�gina separada para el t�tulo
\documentclass[a4paper,11pt,titlepage]{article}

% Carga de paquetes necesarios. OrdenesArticle es un paquete personalizado
\usepackage[spanish]{babel} 
\RequirePackage[T1]{fontenc}
\RequirePackage[ansinew]{inputenx} 
\usepackage[spanish,cap,cont,title,fancy]{OrdenesArticle}
\usepackage{lmodern}
\usepackage{array}
\usepackage{graphicx}
\usepackage{hyperref}
\usepackage{pifont}
\usepackage{listings}
\usepackage[usenames,dvipsnames]{color}
\usepackage{colortbl}
\usepackage{color}
\usepackage{ifthen}
\usepackage{longtable}
\hypersetup{bookmarksopen,bookmarksopenlevel=4,linktocpage,colorlinks,urlcolor=blue,citecolor=blue,
						linkcolor=blue,filecolor=blue,pdfnewwindow,
						pdftitle={Pr�ctica Segundo Cuatrimestre Ingenier�a Software II},
						pdfauthor={Juan Andrada Romero, Juan Gallardo Casero, Jose Domingo L�pez L�pez},
						pdfsubject={Ingenier�a del Software II}}


% Macro para definir una lista personalizada 
\newenvironment{milista}%
{\begin{list}{\textbullet}%
{\settowidth{\labelwidth}{\textbullet} \setlength{\leftmargin}{\dimexpr\labelsep+\labelwidth+5pt}
\setlength{\itemsep}{\dimexpr 0.5ex plus 0.25ex minus 0.25ex}
\setlength{\parsep}{\itemsep}
\setlength{\partopsep}{\itemsep}
\addtolength{\topsep}{-7.5pt}
}}%
{\end{list}}

% Macro para insertar una imagen
%       Uso: \imagen{nombreFichero}{Factor escala}{Caption (leyenda)}{Label (identificador para referenciarla)}
% -------------------------------------------------------------------------------------------------------------
\def\imagen#1#2#3#4{
 \begin{figure}[h]
 \begin{center}
   \scalebox{#2}{\includegraphics{#1}}
 \caption {#3}
 \label{#4}
 \end{center}
 \end{figure}
}

% Macro para insertar un diagrama de un sistema
%       Uso: \diagrama{tipo de diagrama}{nombre del diagrama}
% -------------------------------------------------------------------------------------------------------------
\def\diagrama#1#2{fichero de Visual Paradigm,
 \textbf{Diagrama de}
 \ifthenelse{\equal{#1}{an�lisis}}{\textbf{Clases de An�lisis}}
 {\ifthenelse{\equal{#1}{secuencia}}{\textbf{Secuencia}}
 {\ifthenelse{\equal{#1}{estados}}{\textbf{M�quina de Estados}}
 {\ifthenelse{\equal{#1}{clases}}{\textbf{Clases}}
 {\ifthenelse{\equal{#1}{casosUso}}{\textbf{Casos de Uso}}
 {\ifthenelse{\equal{#1}{colaboracion}}{\textbf{Colaboraci�n}}
 {\ifthenelse{\equal{#1}{paquetes}}{\textbf{Paquetes}}
 {\ifthenelse{\equal{#1}{componentes}}{\textbf{Componentes}}
 {\ifthenelse{\equal{#1}{despliegue}}{\textbf{Despliegue}}{}}}}}}}}} \mbox{-\textgreater}{} \textit{\textbf{#2}}}

% Definicion del "listings" para el lenguaje Java
\lstdefinelanguage{Java}
{
 morecomment = [l]{//}, 
 morecomment = [l]{///},
 morecomment = [s]{/*}{*/},
 morestring = [b]", 
 sensitive = true,
 morekeywords = {package, static, while, switch, break, line, void, String, Object, int, Integer, instanceof, else, if, for, private, return, new, public, class, import, int, boolean, true, false, extends, final, super, protected, abstract, this, do, float, double, null, try, catch, implements}
}

% Configuraci�n del c�digo incluido para el lenguaje Java
\lstset{
  language=Java,
  basicstyle=\footnotesize,
  backgroundcolor=\color{white},
  showspaces=false,
  showstringspaces=false,
  showtabs=false,
  frame=single,
  tabsize=2,
  captionpos=b,
  breaklines=true,
  breakatwhitespace=false,
  escapeinside={\%},
  keywordstyle = \color [rgb]{0,0,1},
  commentstyle = \color [rgb]{0.133,0.545,0.133},
  stringstyle = \color [rgb]{0.627,0.126,0.941}
}

\begin{document}

% En las p�ginas de portada e �ndices, no hay encabezado ni pie de p�gina
\pagestyle{empty} 

% Se incluye la portada
\thispagestyle{empty}
\begin{center}
  {\LARGE UNIVERSIDAD DE CASTILLA-LA MANCHA} \\
  \bigskip
  {\Large ESCUELA SUPERIOR DE INFORM�TICA} \\
  \vspace{28mm}
  \includegraphics[scale=0.45, keepaspectratio]{./imagenes/esi_bw.png} \\
  \vspace{30mm}
  {\LARGE \textbf{Ingenier�a del Software II}} \\
  \vspace{10mm}
  {\Large \textsf{\textsc{Aplicaciones para m�viles con SS.OO Symbian}}} \\
  \vspace{20mm}
  {\large Juan Andrada Romero} \\
  {\large Juan Gallardo Casero} \\
  {\large Jose Domingo L�pez L�pez} \\
  \vspace{9mm}
  {\large \today}
\end{center}
\clearpage

% En las p�ginas de �ndices y prefacio, se utiliza numeraci�n romana
\pagenumbering{Roman}

% Se crea el �ndice
\tableofcontents
% Se pasa p�gina y se a�ade el �ndice de figuras al �ndice principal
%\clearpage\phantomsection
%\addcontentsline{toc}{section}{\listfigurename}
% Se crea el �ndice de figuras
%\listoffigures
% Si se quiere crear un �ndice de tablas se pondr�a: \listoftables

\newpage

% Se ajusta la separaci�n entre p�rrafos
\parskip=10pt

% Se a�ade el prefacio al �ndice
%\clearpage\phantomsection
%\addcontentsline{toc}{section}{Prefacio}

% Comienza el contenido del documento. Se utilizan n�meros ar�bigos y el encabezado y pie de p�gina personalizado
\pagenumbering{arabic}
\pagestyle{fancy}

\section{Conexi�n de la aplicaci�n Web con el servidor}

% Comentar algo, en plan de peque�a introducci�n
% Incluir secciones del diagrama de clases y comentar el proxy 

% Hay que hacer un diagrama parecido al que aparece en el enunciado
% pero donde se vea c�mo hemos hecho la conexi�n con la aplicaci�n web
% (la web usa el servidor RMI del 1er parcial)

\subsection{Cambios realizados en el servidor front-end} \label{cambios}

Ha sido necesario realizar ligeras modificaciones en el sistema desarrollado con anterioridad para poder dar soporte a la funcionalidad que debe tener la aplicaci�n Web y para que dicha aplicaci�n pueda utilizar y comunicarse con el servidor front-end ya desarrollado. 

Estas modificaciones se enumeran a continuaci�n, comentando brevemente el motivo de realizar cada uno de los cambios.

\begin{milista}

	\item La aplicaci�n Web debe permitir iniciar sesi�n a un beneficiario, por lo que es necesario incluir en el servidor front-end una sesion para ellos. Por tanto, la clase \textit{Sesi�n} del servidor front-end ha pasado a ser abstracta y ya no tiene asociado un usuario, para poder generalizar las sesiones de los usuarios y los beneficiarios. \\
	De este modo, la clase abstracta \textit{Sesi�n} tiene tiene dos especializaciones: \textit{SesionUsuario} y \textit{SesionBeneficiario}, cada una de las cuales tiene asociado un objeto del tipo adecuado (usuario o beneficiario).
	Adem�s, se ha incluido el rol de ''Beneficiario'' en la enumeraci�n de roles que ya exist�a en el sistema, renombr�ndose por \textit{Roles}, en lugar de \textit{RolesUsuario}.

	\item Para poder identificar un�vocamente a los clientes que inician sesi�n en la alicaci�n Web sin conocer su rol (m�dico o beneficiario), se ha a�adido el m�todo abstracto ''getNombre()'' a la clase \textit{Sesi�n}, el cual se implementa en la clase \textit{SesionUsuario}, devolviendo el login del usuario, y en la clase \textit{SesionBeneficiario}, devolviendo el NIF del beneficiario.

	\item Debido al cambio en las sesiones, se han tenido que cambiar las operaciones de algunos gestores que acced�an al campo ''usuario'' de la \textit{Sesi�n}, accediendo ahora a dicho campo de la clase \textit{SesionUsurio}. Adem�s, todas las llamadas a los m�todos de la clase \textit{ServidorFrontend} que escrib�an en el log tambi�n se han modificado, para utilizar el nuevo m�todo ''getNombre()'' para identificar al cliente que inicie sesi�n.

	\item En el cliente se han modificado todas las referencias a la enumeraci�n \textit{RolesUsuario} por \textit{Roles}.

	\item El m�todo ''identificar'' del servidor front-end se ha renombrado por ''identificarUsuario'' y se ha creado el m�todo ''identificarBeneficiario'', actualiz�ndose las clases \textit{GestorSesion} y \textit{ServidorFrontend}.

	\item A�adido en la clase \textit{GestorM�dicos} el m�todo para consultar un m�dico a partir de su login, necesario para recuperar el objeto de tipo m�dico cuando un cliente con este rol inicia sesi�n en la aplicaci�n Web. Dicho m�todo tambi�n se ha a�adido en la clase \textit{GestorUsuarios}.
	
	\item Para seguir manteniendo la interfaz \textit{IServidor}, el m�todo para identificar un beneficiario se realiza con el m�todo ''mensajeAuxiliar''. Lo mismo sucede para para consultar un medico por su login.

	\item Las operaciones ''identificarBeneficiario'' y ''consultarPorLogin'' s�lo se han a�adido al proxy de la aplicaci�n Web, pues los clientes de la aplicaci�n de escritorio no utilizan dichos m�todos.

	\item Creado un m�todo en la clase \textit{GestorCitas} para consultar una cita, dado su identificador. Este m�todo se invoca a trav�s del mensaje auxiliar de la interfaz \textit{IServidor}.

	\item Ha sido necesario modificar los permisos de las operaciones que pueden realizar los diferentes roles, para poder permitir realizar dichas operaciones al beneficiario y al m�dico. A continuaci�n, se citan los permisos que se han a�adido al rol de beneficiario y de m�dico: 
	\begin{itemize}
		\item El m�dico debe poder consultar los beneficiarios que tiene asignados, para mostrarlos en la p�gina Web ''darVolante.jsp''. 	
		\item La operaci�n ''ConsultarMedico'' la puede realizar ahora tambi�n un m�dico y un beneficiario, porque se necesita consultar el m�dico receptor para poder emitir un volante. 
		\item Para consultar un medico por su login, se ha creado una operaci�n nueva que se llama ''ConsultarPropioMedico'', que la puede realizar un medico o el administrador del sistema.	
		\item Se ha a�adido un permiso para que un beneficiario pueda consultar sus citas.
		\item A�adido un permiso para que un beneficiario pueda anular una cita.
		\item Modificados los permisos para que un beneficiario pueda consultar las horas libres y ocupadas de un m�dico.
		\item Modificados los permisos para que un beneficiario pueda tramitar una cita, con y sin volante.
	\end{itemize}

\end{milista}

\subsubsection{Cambios referentes a la conexi�n con la base de datos} \label{cambiosHibernate}

Sin duda, la capa de persistencia es la parte del servidor front-end que m�s se ha tenido que modificar para adaptarse a los nuevos requisitos del sistema, puesto que se obligaba a que la persistencia se realizara con \emph{Hibernate} y no a trav�s de un agente SQL (como se comenta en la secci�n \ref{tecnologias}, aunque s�lo se ped�a manejar las citas con \emph{Hibernate}, al final se ha usado dicho framework para la persistencia de todo el sistema).

Como se puede ver en el \diagrama{clases}{Gestor Conexiones BBDD}, para poder utilizar \emph{Hibernate} como gestor de persistencia, ha sido necesario cambiar por completo la clase \emph{GestorConexionesBD}, que era la encargada de centralizar todos los acceso a la base de datos. Ahora, los m�todos de acceso a la base de datos ya no toman como par�metro una sentencia SQL encapsulada en un \emph{ComandoSQL}, sino que en el caso de las inserciones, actualizaciones y eliminaciones se pasa directamente el objeto persistente, mientras que para realizar las consultas se utiliza una nueva clase llamada \emph{ConsultaHibernate}, que agrupa una sentencia escrita en \emph{HQL} (\emph{Hibernate Query Language}) y sus correspondientes par�metros (igual que se hac�a con \emph{ComandoSQL}).

Para adaptarse a estos cambios, se ha tenido que modificar el c�digo de todas las operaciones de las clases de persistencia del servidor front-end (los FPs y la clase \emph{UtilidadesPersistencia}), as� como eliminar aquellas clases que gestionaban tablas que no se usaban directamente, sino a trav�s claves ajenas (como suced�a con \emph{FPDireccion}, \emph{FPTipoMedico} y \emph{FPPeriodoTrabajo}). Las nuevas clases de la capa de persistencia se pueden ver en el \diagrama{clases}{Fabricaciones puras}. Conviene destacar que la clase \emph{HibernateSessionFactory} fue generada autom�ticamente con \emph{MyEclipse} y s�lo se tuvo que modificar para que se pudiera elegir la IP y el puerto en el que se encontraba la base de datos.

Adem�s de modificar las clases de la capa de persistencia, para que las clases del dominio fueran compatibles con \emph{Hibernate}, se tuvieron que cambiar todas las relaciones uno-a-muchos para que usaran colecciones de tipo \emph{Set} y no \emph{Vector} como en el servidor del primer cuatrimestre.

Durante el paso de la persistencia a \emph{Hibernate}, surgieron multitud de problemas debido a que los objetos le�dos de la base de datos se ten�an que enviar a los clientes a trav�s de \emph{RMI}. El framework de \emph{Hibernate} no est� dise�ado para ser ejecutado en un servidor \emph{RMI}, como lo demuestra el hecho de que sustituya las colecciones de objetos que hay en las clases del dominio por clases especiales que permiten una \emph{lazy load}\footnote{T�cnica que consiste en recuperar los objetos de una colecci�n cuando se accede por primera vez a ella.} pero que no son serializables y, por tanto, no se pueden transmitir mediante \emph{RMI}.

Para solucionar estos y otros problemas relacionados con las referencias de los objetos persistentes, se ha tenido que hacer un uso exhaustivo de los m�todos \emph{clone} de las clases del dominio, con el fin de convertir las colecciones especiales de \emph{Hibernate} en listas serializables; y a�adir nuevos m�todos al \emph{GestorConexionesBD}, como \emph{iniciarTransaccion}, \emph{terminarTransaccion} y \emph{borrarCache}, para poder realizar las operaciones m�s complejas de modificaci�n de usuarios y beneficiarios.

Finalmente, aunque se ha mantenido la estructura de clases que permitir�a al servidor front-end funcionar junto con el servidor de respaldo, tras muchas pruebas no se ha conseguido que en ambos servidores se gestione la persistencia a trav�s de \emph{Hibernate} (este framework tampoco est� dise�ado para ello). Por este motivo, en el nuevo servidor front-end el uso del servidor de respaldo est� desactivado por defecto y no se permite activarlo.

\clearpage
\section{Tecnolog�as utilizadas}
\label{tecnologias}

En este apartado se comentan las tecnolog�as que se han empleado para el desarrollo de la aplicaci�n web y la adaptaci�n del servidor front-end, resaltando aquellos aspectos que suponen una mejora con respecto al enunciado original de la pr�ctica.

\begin{milista}

	\item Toda la persistencia del servidor front-end se gestiona a trav�s de \textbf{\emph{Hibernate}}. En el enunciado se ped�a como condici�n m�nima que se utilizara \emph{Hibernate} para la persistencia de las citas, pero al final se ha conseguido adaptar el servidor front-end para que todos los accesos a la base de datos se hagan mediante este framework. Por lo tanto, en la nueva versi�n del servidor no se emplea ning�n agente de base de datos, y tampoco se hace uso del lenguaje SQL.
	
	\item La conexi�n entre la aplicaci�n web y el servidor front-end se realiza mediante \textbf{\emph{RMI}}, utilizando la misma interfaz que la aplicaci�n de escritorio desarrollada en el primer cuatrimestre para acceder a las funcionalidades del servidor.
	
	% Comentar el uso de Ajax, JSP, CSS (que no se pide en la pr�ctica, es una mejora), Struts, validadores y conversores (si al final los acabamos usando)
	
\end{milista}

\clearpage
\section{Caracter�sticas de la aplicaci�n Web}

Para evitar que los usuarios del sistema traten de acceder a p�ginas de la aplicaci�n de un modo indebido, se controla que dichas p�ginas sean accedidas desde el \textit{index} y nunca a trav�s de la barra de direcciones del navegador.

Adem�s, una vez configurada la aplicaci�n Web en el servidor, ser� necesario colocar el fichero \textit{struts.xml} que se encuentra en el directorio \textit{WebContent/WEB-INF/} en \textit{WebContent/WEB-INF/classes/}. Inicialmente no se distribuye dicho fichero en su lugar correcto ya cada vez que se compila la aplicaci�n ser� eliminado. Tenga en cuenta que para un correcto funcionamiento, es necesario que se encuentre activo el servidor front-end proporcionado con el sistema.

Por �ltimo, la configuraci�n del servidor de bases de datos, as� como los usuarios y sus permisos, se har� del mismo modo que en la pr�ctica del primer parcial.
\clearpage

% A�adimos la bibliografia al �ndice
\clearpage\phantomsection
\addcontentsline{toc}{section}{Referencias}

% Bibliograf�a. En este caso se usa BibTeX
\bibliographystyle{plain}
\pagestyle{plain} 
\bibliography{Bibliografia}

\end{document}
