\section{Tecnolog�as utilizadas}
\label{tecnologias}

En este apartado se comentan las tecnolog�as que se han empleado para el desarrollo de la aplicaci�n web y la adaptaci�n del servidor front-end, resaltando aquellos aspectos que suponen una mejora con respecto al enunciado original de la pr�ctica.

\begin{milista}

	\item Toda la persistencia del servidor front-end se gestiona a trav�s de \textbf{\emph{Hibernate}}. En el enunciado se ped�a como condici�n m�nima que se utilizara \emph{Hibernate} para la persistencia de las citas, pero al final se ha conseguido adaptar el servidor front-end para que todos los accesos a la base de datos se hagan mediante este framework. Por lo tanto, en la nueva versi�n del servidor no se emplea ning�n agente de base de datos, y tampoco se hace uso del lenguaje SQL.
	
	\item La conexi�n entre la aplicaci�n web y el servidor front-end se realiza mediante \textbf{\emph{RMI}}, utilizando la misma interfaz que la aplicaci�n de escritorio desarrollada en el primer cuatrimestre para acceder a las funcionalidades del servidor.
	
	% Comentar el uso de Ajax, JSP, CSS (que no se pide en la pr�ctica, es una mejora), Struts, validadores y conversores (si al final los acabamos usando)
	
\end{milista}
