En los diferentes puntos de esta secci�n se van a detallar las decisiones de dise�o que se han tenido en cuenta a la hora de dise�ar y desarrollar los tres sistemas: unidad de citaci�n o cliente; servidor front-end y servidor de respaldo.

\subsubsection{Arquitectura multicapa}
Cabe destacar que todos los sistemas se han desarrollado siguiendo una arquitectura multicapa. De este modo, las capas con las que cuenta cada sistema son las siguientes: 

\begin{milista}
	\item \textbf{Unidad de citaci�n}: comunicaciones, dominio y presentaci�n.
	\item \textbf{Servidor front-end}: comunicaciones, dominio, persistencia y presentaci�n.
	\item \textbf{Servidor de respaldo}: comunicaciones, dominio, persistencia y presentaci�n.
\end{milista}

Utilizando este enfoque multicapa, se sigue el principio de m�nimo acoplamiento y m�xima cohesi�n, desacoplando los elementos de una capa de los de otra. De este modo, se facilita el mantenimiento y extensibilidad del sistema, pues cuando se realice el cambio en alg�n elemento de una capa, el resto de capas no se ver�n afectadas.

En los siguientes apartados se comentar�n las decisiones de dise�o que se han tenido en cuenta para cada una de las capas de cada sistema desarrollado.

\subsubsection{Servidor Front-end}
\paragraph{Capa de dominio} \label{frontendDominio}

Una de las primeras decisiones tomadas fue la de dividir esta capa en un paquete de conocimiento y en otro de control. En el primero de ellos se encuentran todas las clases del modelo de dominio (adem�s de algunas enumeraciones e interfaces con constantes), que son aquellas clases cuya �nica responsabilidad es la de mantener informaci�n. \\
\indent Por otro lado, en el paquete de control se han colocado todas las clases que se encargan de gestionar las clases de conocimiento y las comunicaciones con la capa de persistencia, presentaci�n y con la unidad de citaci�n (a trav�s de la capa de comunicaciones). Sin embargo, para cumplir con el principio de m�nimo acoplamiento y m�xima cohesi�n, estos \textit{gestores} agrupan responsabilidades que est�n relacionadas con la gesti�n de una clase del modelo de conocimiento. Por ejemplo, para gestionar la clase de conocimiento \textit{Beneficiario}, se ha creado la clase \textit{GestorBeneficiarios}, cuyas responsabilidades son consultar un beneficiario, crearlo, etc. \\
\indent Se puede observar la divisi�n de esta capa y las relaciones con otras capas en el \diagrama{frontend}{paquetes}{Divisi�n en Capas}.

Adem�s, estos gestores delegan la persistencia de las clases de conocimiento a las diferentes \textit{Fabricaciones Puras} de la capa de persistencia, evitando as� que las clases del modelo de dominio est�n acopladas con la capa de persistencia, lo que se puede considerar como un patr�n \textbf{Indirecci�n}. \\
\indent Todos los gestores, y las relaciones existentes entre ellos, se pueden apreciar en el \diagrama{frontend}{clases}{Gestores}.

En lo que se refiere a los tipos de usuarios que pueden utilizar el sistema, se decidi� utilizar las siguientes clases y jerarqu�a de herencia: existe una clase abstracta llamada \textit{Usuario} de la que heredan las clases \textit{M�dico}, \textit{Administrador} y \textit{Citador}. La clase \textit{Usuario} es abstracta ya que no se van a instanciar objetos directamente de esa clase, sino de una de sus subclases, que representan los diferentes usuarios que pueden existir en el sistema, seg�n la especificaci�n de requisitos (ver secci�n \ref{requisitosIniciales}). Adem�s, la soluci�n propuesta facilita la extensibilidad futura del sistema, ya que el rol de cada usuario se define mediante la enumeraci�n \textit{RolesUsuario}, de tal modo que cada subclase de la clase \textit{Usuario} redefinir� el m�todo abstracto \textit{getRol()}, devolviendo el valor correspondiente a su rol en la enumeraci�n. Por tanto, para a�adir un nuevo rol de usuario, basta con a�adir una nueva clase que herede de la superclase y un nuevo rol a la enumeraci�n.\\
\indent Tambi�n, seg�n la especificaci�n de requisitos, la clase \textit{M�dico} hereda de \textit{Usuario} porque los m�dicos deben utilizar la Unidad de Citaci�n para consultar sus propias citas, emitir volantes y consultar beneficiarios (ver Tabla \ref{tb:operacionesInterfaz}), haciendo necesario que est�n dados de alta en el sistema y dispongan de su nombre de usuario y contrase�a correspondientes. \\
\indent Por otra parte, la clase \textit{Administrador} hereda de la clase \textit{Citador} porque un administrador puede hacer todas las operaciones de un citador m�s las suyas propias (ver Tabla \ref{tb:operacionesInterfaz}). \\
\indent Esta jerarqu�a de herencia se puede observar en el \diagrama{frontend}{clases}{Gestor Usuarios}.

Por otra parte, para aumentar la seguridad del sistema, se decidi� encriptar la contrase�a de los usuarios, utilizando para ello el algoritmo SHA1. 
Algunas de las razones para utilizar una encriptaci�n por c�digo y no delegar esta responsabilidad en el sistema gestor de base de datos, son las siguientes: 
\begin{enumerate}
	\item Si se quiere cambiar la encriptaci�n a una m�s segura, no har�a falta m�s que cambiar el m�todo que encripta la contrase�a.
	\item Puede que otros SGBD que no sean MySQL no tengan encriptaci�n incorporada.
	\item El n�mero de encriptaciones que incorpora un SGBD es limitado.
\end{enumerate}

Siguiendo con el tema de la herencia, se plante� la posibilidad de hacer una clase \textit{Persona} de la que heredan \textit{Usuario} y \textit{Beneficiario}, pero esta alternativa ha sido rechazada debido a que al aplicar el patr�n de persistencia \textbf{1 Camino de Herencia, 1 tabla}, el resultado obtenido ser�a el mismo que el que se ha obtenido. Adem�s, dicha herencia no estar�a del todo justificada, ya que �nicamente se especializar�an las clases definiendo algunos atributos diferentes a los de la superclase, pero no m�todos ni comportamientos distintos, como ocurre en el caso de la herencia de usuarios.

Otra decisi�n de dise�o tomada ha sido utilizar el patr�n \textbf{Estado} para representar los diferentes tipos de m�dicos. Dicho patr�n facilita la futura extensibilidad del sistema, debido a que un m�dico puede pertenecer a distintos tipos (por ejemplo, cabecera y especialista) y se podr�a cambiar su tipo en tiempo de ejecuci�n para ejecutar unas u otras operaciones, dependiendo del tipo de m�dico. \\
\indent No obstante, en lo que se refiere a este aspecto (cambio de tipo en tiempo de ejecuci�n), el uso de este patr�n no est� totalmente justificado, pues actualmente no se permite que un m�dico pueda ser de dos tipos. \\
\indent La implementaci�n de este patr�n puede observarse en el \diagrama{frontend}{clases}{Gestor Usuarios}.

Referente tambi�n a los m�dicos, en la especificaci�n de requisitos (ver secci�n \ref{requisitosIniciales}) se indica que los m�dicos tienen un calendario laboral, en el cu�l se definen los d�as y las horas que trabajan. En el sistema, �sto se ha representado mediante el concepto \textit{conjunto de periodos de trabajos}, donde un \textit{periodo de trabajo} es una clase que define el d�a de la semana y el intervalo de horas consecutivas que se trabaja ese d�a (hora de inicio y hora de fin del intervalo). De este modo, se permite que un m�dico pueda definir un calendario laboral variable, trabajando, por ejemplo, el Lunes de 9:00 a 14:00 y de 17:00 a 21:00. \\
\indent Hay que resaltar que se crea un nuevo periodo de trabajo cuando las horas no son consecutivas dentro del intervalo seleccionado. As�, en el ejemplo anterior se crearian dos periodos diferentes para el mismo d�a).

En cuanto a lo que se refiere a la interfaz remota que debe ofrecer el servidor front-end, se ha decidido implementar una clase \textit{ServidorFrontend} que es la que implementa dicha interfaz y, a la vez, es un patr�n \textbf{Fachada} y \textbf{Singleton}, de tal modo que esta fachada �nicamente recibe las peticiones del cliente y las delega en los distintos gestores que forman parte de la l�gica del sistema. \\
\indent Esto se puede apreciar en el \diagrama{frontend}{clases}{Servidor Front-end}.

Otro de los requisitos del sistema consiste en que al realizar determinadas operaciones desde alguno de los clientes conectados al servidor, el resto de clientes deben actualizar sus ventanas, si procede, para reflejar dichos cambios. Para esto, se implementa un patr�n \textbf{Observador}, donde el \textit{Gestor de Sesiones} del servidor front-end adquiere el rol de observado (\textit{''subject''}), ya que mantiene la lista de clientes activos (que son los observadores) y notifica los cambios a cada uno de ellos. Los observadores, a su vez, implementan una interfaz \textit{ICliente}, sobre la cu�l realiza las llamadas el \textit{Gestor de Sesiones} para notificar los cambios. De este modo, si el cliente cambia su implementaci�n del m�todo que actualiza los cambios en la interfaz gr�fica, el servidor sigue siendo independiente, ya que est� programado contra la interfaz \textit{ICliente} (ver \diagrama{frontend}{clases}{Gestor Sesiones}).

Continuando con el \textit{Gestor de Sesiones}, otra decisi�n de dise�o que se ha tomado ha sido que si se elimina el usuario que tiene una sesi�n iniciada en el sistema, se cierra esa sesi�n y se actualiza al cliente oportuno. Tambi�n, si un usuario inicia una sesi�n que ya est� activa desde otro terminal, se cierra la sesi�n activa y se abre la nueva sesi�n. \\
\indent Para terminar, hay que se�alar que si se desconecta el servidor front-end y existen clientes conectados a �l, se cierran todas las sesiones activas y se informa a dichos clientes.
\paragraph{Capa de presentaci�n} \label{presentacion-front-end}

En el servidor front-end, la capa de presentaci�n no tiene demasiada
relevancia, ya que s�lo se encarga de mostrar la ventana de estado del servidor
al usuario y permitirle conectar, desconectar y configurar el servidor
front-end. Se tom� la decisi�n de mostrar los mensajes de estado en el servidor
(y no simplemente guardarlos en la base de datos) para que el administrador
encargado de mantener el servidor tenga realimentaci�n sobre las operaciones
que realiza el sistema mientras est� en ejecuci�n.

La clase encargada de mostrar los mensajes de estado en la ventana del
servidor front-end es \textit{ConexionLogVentana}, una clase del paquete de
comunicaciones a la que se hace referencia en la secci�n
\ref{comunicaciones-front-end}. En esta clase se ha aplicado el patr�n
\textbf{Observador} para permitir que los mensajes se puedan mostrar en m�s
de una ventana a la vez, sin necesidad de crear varias instancias de
\textit{ConexionLogVentana}.

As� pues, como se observa en el \diagrama{frontend}{clases}{Gestor Conexiones
Log}, la clase \textit{ConexionLogVentana} tiene asociado cero, uno o m�s
instancias de la interfaz \textit{IVentanaEstado}, a las que reenv�a todos los
mensajes que le llegan. En la implementaci�n final del sistema, s�lo hay una
clase que implementa esta interfaz, que es la ventana principal del servidor
front-end, \textit{JFServidorFrontend}, pero el uso del patr�n
\textbf{Observador} facilita notablemente la extensibilidad del sistema.

\paragraph{Capa de persistencia} \label{persistencia-front-end}

En esta capa, entre otras decisiones de dise�o que se comentar�n a continuaci�n, se hace uso de los patrones fabricaci�n pura, comando, agente y singleton. Para una mejor comprensi�n se recomienda ver el \diagrama{cliente}{clases}{Gestor Conexiones BBDD}.

Para abstraer al programador de las caracter�sticas espec�ficas del lenguaje de programaci�n empleado (Java) al realizar operaciones en la base de datos se utiliza el patr�n \textbf{Comando} o \textbf{Command}, permiti�ndo encapsular en una clase de m�s alto nivel la construcci�n de la sentencia que se ejecutar� en la base de datos. La clase \textit{ComandoSQL} tiene dos especializaciones: \textit{ComandoSQLProcedimiento} y \textit{ComandoSQLSentencia}. De este modo, el programador �nicamente deber� pasar la sentencia SQL y la lista de par�metros de n�mero variable que hay que incrustar en ella, abstray�ndole por completo de la creaci�n de sentencias espec�ficas del lenguaje (\textit{PreparedStatement} y \textit{CallabeStatement}).

Para proveer de un acceso al SGBD (Sistema Gestor de Bases de Datos) se ha implementado una clase \textbf{Agente} o \textbf{Broker} junto con el patr�n \textbf{Singleton}. El agente es el que inicializa, mantiene y cierra la conexi�n con el SGBD, manteni�ndose una sola instancia de �ste debido a que es singleton. De este modo se evita se evita la creaci�n de m�ltiples instancias del agente para no saturar el SGBD. Adem�s, se encarga de ejecutar las sentencias encapsuladas en el ComandoSQL. Por otro lado, n�tese que implementa las acciones \textit{commit()} y \textit{rollback()}.

En referencia a la persistencia de las clases de conocimiento persistentes, se han tenido en cuenta dos alternativas: \textbf{patr�n experto} y \textbf{patr�n de fabricaci�n pura}. Para ser fieles con el \underline{principio de m�xima coherencia y bajo acomplamiento}, se ha descartado el patr�n experto ya que se acoplar�an las clases de conocimiento con la persistencia, violando as� dicho principio. Por tanto se ha utilizando el patr�n de fabricaci�n pura implementando clases que tienen como �nica responsabilidad las operaciones CRUD (Create, Read, Update, Delete) de las clases persistentes, de tal modo que si se cambia el SGDB s�lo se tendr�an que cambiar el modo en que dichas clases definen las sentencias SQL, ya que la creaci�n est� delegada en el ComandoSQL. Adem�s, cada una de estas clases de fabricaci�n pura se encargan de acceder exclusivamente a una tabla de la base de datos, reduciendo as� al m�ximo el acoplamiento entre clases de persistencia.
\subsubsection{Servidor front-end}

\paragraph{Capa de comunicaciones}

\subsubsection{Servidor de Respaldo}
\paragraph{Capa de dominio}

En el sistema del servidor de respaldo, la capa de dominio s�lo contiene clases relacionadas con el control de este sistema, careciendo de clases de conocimiento.

Una de las clases de este paquete de control, llamada \textit{ControladorRespaldo}, se centra en iniciar y detener el servidor de respaldo, as� como de mostrar la interfaz gr�fica de la capa de presentaci�n.

Por otra parte, la clase \textit{ServidorRespaldo} es la clase que implementa la interfaz \textit{IServidorRespaldo}, que, entre otros, tiene los m�todos necesarios para actualizar la interfaz gr�fica con las diferentes peticiones que se van recibiendo en este servidor, y m�todos para ejecutar en el SGBD las sentencias SQL que recibe por parte del servidor front-end (a trav�s de la capa de comunicaciones), encapsuladas en objetos de la clase \textit{ComandoSQL}.

Estas comunicaciones entre capas se pueden observar en el \diagrama{respaldo}{paquetes}{Divisi�n en Capas}.

\paragraph{Capa de presentaci�n}

Todas las decisiones que se tomaron para crear la capa de presentaci�n del
servidor front-end (ver secci�n \ref{presentacion-front-end}) son aplicables
para el servidor de respaldo, as� que no las comentaremos de nuevo en este
apartado.

\paragraph{Capa de persistencia}

Dado que el servidor de respaldo se limita a construir una r�plica exacta de la base de datos del SGBD principal en la base de datos del SGBD de respaldo, este sistema carece de cualquier clase de conocimiento. Debido a esto, tampoco contiene las clase de persistencia que se comentaban en la capa de persistencia del servidor front-end (ver \ref{persistencia-front-end}), ya que su �nica funci�n es ejecutar las sentencias que le env�a el servidor front-end.

Dicho esto, n�tese que la capa de persistencia s�lo contiene la clase que gestiona la conexiones al SGDB. Esta clase, al igual que en el servidor front-end (ver \ref{persistencia-front-end}), se ha dise�ado por medio de los patrones \textbf{Agente} y \textbf{Singleton}.

\subsubsection{Servidor de respaldo}

\paragraph{Capa de comunicaciones}

\subsubsection{Unidad de Citaci�n}
\paragraph{Capa de dominio}

En el sistema cliente, la capa de dominio s�lo contiene clases relacionadas con el control de este sistema, careciendo de clases de conocimiento, ya que todas esas clases se encuentran en el lado del servidor. De este modo, el cliente se limita a realizar peticiones al servidor (utilizando la capa de comunicaciones), obtener su respuesta y gestionar la capa de presentaci�n seg�n sea necesario. 

Una de las clases de este paquete de control, llamada \textit{ControladorCliente}, se centra en controlar los diferentes elementos de la capa de presentaci�n y de comunicarse con el servidor front-end, utilizando la capa de comunicaciones.

Por otra parte, la clase \textit{Cliente} es la clase que implementa la interfaz \textit{ICliente}, que, entre otros m�todos, tiene un m�todo para que el controlador actualice la interfaz gr�fica con las respuestas que obtiene del servidor, cuando otro cliente hace una petici�n. 

Estas comunicaciones entre capas se pueden observar en el \diagrama{cliente}{paquetes}{Divisi�n en Capas}.


\paragraph{Capa de presentaci�n}

De las tres aplicaciones que forman el sistema construido, la unidad de
citaci�n es aquella en la que la capa de presentaci�n es m�s importante,
debido a que es la que utilizar� el usuario final para manipular el sistema de
salud.

El desarrollo de la interfaz gr�fica del cliente se ha basado en la divisi�n
del sistema que se hizo en la gesti�n de requisitos (ver secci�n
\ref{requisitos-detallados}). De esta forma, a cada gestor identificado
(beneficiarios, usuarios, citas, volantes y sustituciones) se le ha asignado
una pesta�a en la interfaz gr�fica, dentro de la cual se agrupan los paneles
de todas las operaciones relacionadas. Por sencillez para el usuario, las
operaciones de consulta, modificaci�n y eliminaci�n se han unido en una �nica
operaci�n, por lo que para eliminar o modificar un usuario primero se debe
consultar.

Un detalle interesante que conviene destacar de las operaciones disponibles
para cada usuario es que �stas no est�n fijadas en el c�digo de la unidad de
citaci�n, sino que cuando un usuario inicia sesi�n, pide las operaciones
que puede realizar al servidor front-end y, en funci�n de la respuesta
recibida, se muestran unas u otras operaciones. La gran ventaja de esta forma
de determinar las operaciones para un usuario es que, si se crea un nuevo tipo
de usuario o cambian las operaciones asignadas a un tipo de usuario existente,
los cambios que habr�a que hacer en la unidad de citaci�n ser�an m�nimos.

A la hora de implementar la interfaz, se tom� la decisi�n de crear un panel
(una clase) por cada operaci�n que el usuario puede realizar con el sistema.
Por tanto, hay un panel dedicado a registrar beneficiarios, otro a
consultar/modificar/eliminar usuarios, otro diferente a tramitar citas, etc.
Esta divisi�n ha permitido reutilizar paneles en otros paneles, aprovechando
as� la potencia de la orientaci�n a objetos. Por ejemplo, en el panel de
tramitaci�n de citas, \textit{JPCitaTramitar}, se utiliza el panel de
consulta de beneficiarios, \textit{JPBeneficiarioConsultar}, para elegir el
beneficiario que quiere pedir cita. En el \diagrama{cliente}{clases}{Ventana
Principal} se puede ver con detalle la jerarqu�a de paneles usada en la
interfaz gr�fica, as� como los paneles que est�n formados de otros paneles.

Debido a la gran cantidad de clases que ten�a el paquete de presentaci�n, se
decidieron separar todas aquellas clases que no eran ventanas o paneles de la
interfaz gr�fica y moverlas al subpaquete \textit{presentacion.auxiliar}. As�
pues, en este paquete se encuentran implementaciones personalizadas de algunos
componentes Java, definiciones de eventos usados en algunos paneles y clases
de utilidades con m�todos que se reutilizan en varios paneles.

Una clase especialmente importante en este subpaquete es \textit{Validacion},
que es donde se encuentran todos los m�todos encargados de validar los campos
introducidos en el usuario. Como se puede ver en los diagramas de secuencia del
cliente (por ejemplo, en el \diagrama{cliente}{secuencia}{Registrar M�dico}),
siempre que el usuario quiere realizar una operaci�n con la unidad de
citaci�n, se llama a la clase \textit{Validacion} para comprobar que los datos
de la operaci�n tienen el formato correcto, con el fin de evitar enviar al
servidor peticiones que se sabe que son incorrectas.


\paragraph{Capa de persistencia}
La unidad de citaci�n representa el cliente dentro de una arquitectura cliente-servidor, por lo que no debe realizar ninguna acci�n relacionada con la persistencia, sino que estas operaciones debe delegarlas al servidor. 
\subsubsection{Unidad de Citaci�n}

\paragraph{Capa de comunicaciones}
