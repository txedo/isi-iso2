\subsection{Vista L�gica}

En los diferentes puntos de esta secci�n se van a detallar las decisiones de dise�o que se han tenido en cuenta a la hora de dise�ar y desarrollar los tres sistemas: unidad de citaci�n o cliente; servidor front-end y servidor de respaldo.

\subsubsection{Arquitectura cliente-servidor}

\subsubsection{Arquitectura multicapa}
Cabe destacar que todos los sistemas se han desarrollado siguiendo una arquitectura multicapa. De este modo, las capas con las que cuenta cada sistema son las siguientes: 

\begin{milista}
	\item \textbf{Unidad de citaci�n}: comunicaciones, dominio y presentaci�n. (ver \diagrama{cliente}{paquetes}{})
	\item \textbf{Servidor front-end}: comunicaciones, dominio, persistencia y presentaci�n. (ver \diagrama{frontend}{paquetes}{})
	\item \textbf{Servidor de respaldo}: comunicaciones, dominio, persistencia y presentaci�n. (ver \diagrama{respaldo}{paquetes}{})
\end{milista}

Utilizando este enfoque multicapa, se sigue el principio de m�nimo acoplamiento y m�xima coherencia, desacoplando los elementos de una capa de los de otra. De este modo, se facilita el mantenimiento y extensibilidad del sistema, pues cuando se realice el cambio en alg�n elemento de una capa, el resto de capas no se ver�n afectadas.

-------- �� HABLAR DE COMO SE COMUNICAN LAS CAPAS ?? -------------------

En los siguientes apartados se comentar�n las decisiones de dise�o que se han tenido en cuenta para cada una de las capas de cada sistema desarrollado.

\subsubsection{Unidad de Citaci�n}

\paragraph{Capa de comunicaciones}

\paragraph{Capa de dominio}
En el sistema cliente, la capa de dominio s�lo contiene clases para el control de este sistema, que se encargan de mostrar los diferentes elementos de la capa de presentaci�n y de comunicarse con el servidor front-end, utilizando para ello la capa de comunicaciones. 

\paragraph{Capa de presentaci�n}

\paragraph{Capa de persistencia}
La unidad de citaci�n representa el cliente dentro de una arquitectura cliente-servidor, por lo que no debe realizar ninguna acci�n relacionada con la persistencia, sino que estas operaciones debe delegarlas al servidor. 


\subsubsection{Servidor front-end}

\paragraph{Capa de comunicaciones}

\paragraph{Capa de dominio}

\paragraph{Capa de presentaci�n}

\paragraph{Capa de persistencia}


\subsubsection{Servidor de respaldo}

\paragraph{Capa de comunicaciones}

\paragraph{Capa de dominio}

\paragraph{Capa de presentaci�n}

\paragraph{Capa de persistencia}
