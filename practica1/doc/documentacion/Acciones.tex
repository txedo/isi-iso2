\section{��EN QUE SECCION PONEMOS ESTA PARTE??}

\subsection{Operaciones permitidas en el sistema}

En la tabla \ref{tb:operacionesInterfaz} se muestran las acciones que un usuario puede realizar con el sistema (interactuando con su interfaz gr�fica), junto con el rol que es necesario para poder llevarlas a cabo. Se�alar que estas operaciones se corresponden con las distintas funcionalidades que el sistema ofrece al usuario, es decir, se corresponden con los distintos casos de uso del cliente (ver \diagrama{cliente}{casosUso}{Casos de Uso}).

\begin{longtable}{| p{6.3cm} | c | c | c |}
\hline
\multicolumn{1}{|>{\columncolor[rgb]{0.8, 0.8, 0.8}}c|}{\textbf{OPERACI�N}} & \multicolumn{1}{|>{\columncolor[rgb]{0.8, 0.8, 0.8}}c|}{\textbf{ADMINISTRADOR}} & \multicolumn{1}{|>{\columncolor[rgb]{0.8, 0.8, 0.8}}c|}{\textbf{CITADOR}} & \multicolumn{1}{|>{\columncolor[rgb]{0.8, 0.8, 0.8}}c|}{\textbf{M�DICO}} \\
\hline
\centering{Registrar beneficiario} & X & X & \\
\hline
\centering{Consultar beneficiario} & X & X & X \\
\hline
\centering{Modificar beneficiario} & X & X & \\
\hline
\centering{Eliminar beneficiario} & X & X & \\
\hline
\centering{Registrar usuario} & X &  & \\
\hline
\centering{Consultar usuario} & X &  & \\
\hline
\centering{Modificar usuario} & X &  & \\
\hline
\centering{Eliminar usuario} & X &  & \\
\hline
\centering{Registrar m�dico} & X &  & \\
\hline
\centering{Consultar m�dico} & X & X & \\
\hline
\centering{Modificar m�dico} & X &  & \\
\hline
\centering{Eliminar m�dico} & X &  & \\
\hline
\centering{Tramitar cita} & X & X & \\
\hline
\centering{Tramitar cita a partir de un volante} & X & X & \\
\hline
\centering{Consultar citas de un beneficiario} & X & X & \\
\hline
\centering{Consultar citas de un m�dico} & X & X & \\
\hline
\centering{Anular cita} & X & X & \\
\hline
\centering{Consultar citas propias de un m�dico} & & & X \\
\hline
\centering{Emitir volante} & & & X \\
\hline
\centering{Establecer sustituto} & X &  & \\
\hline
\caption{Operaciones que un usuario puede realizar en el sistema}
\label{tb:operacionesInterfaz}
\end{longtable}


Estas operaciones se van a corresponder con los servicios que proporciona el servidor front-end en su interfaz. Sin embargo, algunas de las operaciones anteriores se apoya en la ejecuci�n de otras acciones. Por ejemplo, para que un administrador pueda establecer una sustituci�n, primero debe buscar una lista de posibles sustitutos y elegir uno de ellos como sustituto de un m�dico; para que un m�dico pueda emitir un volante para un especialista, debe seleccionar un especialista de una lista; se deben calcular las horas a las que trabaja un m�dico para poder pedir cita, etc. Por esta raz�n, se necesitan otras operaciones adem�s de las mostradas en la tabla \ref{tb:operacionesInterfaz}, que se van a corresponder con la operaci�n \textit{mensajeAuxiliar} del servidor front-end. 

En la tabla \ref{tb:operaciones} se muestran, el resto de operaciones que existen en el sistema. As�, en dicha tabla se resumen aquellas operaciones auxiliares, junto con la operaci�n de la tabla anterior a la que dan soporte. 

\begin{longtable}{| p{7cm} | c |}
\hline
\multicolumn{1}{|>{\columncolor[rgb]{0.8, 0.8, 0.8}}c|}{\textbf{ACCI�N}} & \multicolumn{1}{|>{\columncolor[rgb]{0.8, 0.8, 0.8}}c|}{\textbf{DA SOPORTE A}} \\
\hline
\centering{Consultar beneficiarios asignados a un m�dico} & ... \\
\hline
\centering{Corresponde NIF a un usuario} & ...\\
\hline
\centering{Consultar usuario que ha iniciado sesi�n} & ...\\
\hline
\centering{Consultar centros de salud}& ...\\
\hline
\centering{Registrar m�dico} & ...\\
\hline
\centering{Consultar m�dico} & ...\\
\hline
\centering{Modificar m�dico} & ...\\
\hline
\centering{Eliminar m�dico} & ...\\
\hline
\centering{Consultar m�dicos de un tipo determinado} & ...\\
\hline
\centering{Consultar posibles sustitutos} & ...\\
\hline
\centering{Consultar m�dico que atiende una cita} & ...\\
\hline
\centering{Consultar horario de un m�dico} & ...\\
\hline
\centering{Consultar horas de citas asignadas a un m�dico} & ...\\
\hline
\centering{Consultar d�as completos de un m�dico} & ...\\
\hline
\centering{Consultar citas en una fecha de un m�dico} & ...\\
\hline
\centering{Consultar Volante} & ...\\
\hline
\caption{Operaciones auxiliares}
\label{tb:operaciones}
\end{longtable}


----------------- COMPLETAR COMENTANDO UN POCO CADA UNA DE LAS OPERACIONES AUXILIARES (CON LA LISTA QUE HICIMOS, TXEDO)------------------
\begin{milista}
	\item \textbf{OPERACION TAL}: HACE TAL Y SE USA EN TAL.
	\item \textbf{......}
\end{milista}