\subsection{Operaciones permitidas en el sistema}

En la tabla \ref{tb:operacionesInterfaz} se muestran las acciones que un usuario puede realizar con el sistema (interactuando con su interfaz gr�fica), junto con el rol que es necesario para poder llevarlas a cabo. Se�alar que estas operaciones se corresponden con las distintas funcionalidades que el sistema ofrece al usuario, es decir, se corresponden con los distintos casos de uso del cliente (ver \diagrama{cliente}{casosUso}{Casos de Uso}).

\begin{longtable}{| p{8cm} | p{8cm} |}
\hline
\multicolumn{1}{|>{\columncolor[rgb]{0.8, 0.8, 0.8}}c|}{\textbf{ACCI�N}} & \multicolumn{1}{|>{\columncolor[rgb]{0.8, 0.8, 0.8}}c|}{\textbf{ROL}} \\
\hline
Registrar beneficiario & Administrador, Citador\\
\hline
Consultar beneficiario & Administrador, Citador, M�dico\\
\hline
Modificar beneficiario & Administrador, Citador\\
\hline
Eliminar beneficiario & Administrador, Citador\\
\hline
Registrar usuario & Administrador\\
\hline
Consultar usuario & Administrador\\
\hline
Modificar usuario & Administrador\\
\hline
Eliminar usuario & Administrador\\
\hline
Tramitar cita & Administrador, Citador\\
\hline
Tramitar cita a partir de un volante & Administrador, Citador\\
\hline
Consultar citas de un beneficiario & Administrador, Citador\\
\hline
Consultar citas de un m�dico & Administrador, Citador\\
\hline
Anular cita & Administrador, Citador\\
\hline
Consultar citas propias de un m�dico & M�dico\\
\hline
Emitir volante & M�dico\\
\hline
Establecer sustituto & Administrador\\
\hline
\caption{Operaciones que un usuario puede realizar con el sistema}
\label{tb:operacionesInterfaz}
\end{longtable}

Estas operaciones se van a corresponder con los servicios que proporciona el servidor front-end en su interfaz. Sin embargo, algunas de las operaciones anteriores se apoya en la ejecuci�n de otras acciones. Por ejemplo, para que un administrador pueda establecer una sustituci�n, primero debe buscar una lista de posibles sustitutos y elegir uno de ellos como sustituto de un m�dico; para que un m�dico pueda emitir un volante para un especialista, debe seleccionar un especialista de una lista; se deben calcular las horas a las que trabaja un m�dico para poder cita, etc. Por esta raz�n, se necesitan otras operaciones adem�s de las mostradas en la tabla \ref{tb:operacionesInterfaz}, que se van a corresponder con la operaci�n \textit{mensajeAuxiliar} del servidor front-end. 

En la tabla \ref{tb:operaciones} se muestran, adem�s de las operaciones anteriores, el resto de operaciones que existen en el sistema. As�, dicha tabla resume todas las operaciones que se pueden realizar en el sistema, junto con el rol necesario para poder ejecutarlas. Adem�s, en esa tabla se resaltan aquellas operaciones que no aparecen en la tabla \ref{tb:operacionesInterfaz} y que, por tanto, son operaciones auxiliares.

\begin{longtable}{| p{8cm} | p{8cm} |}
\hline
\multicolumn{1}{|>{\columncolor[rgb]{0.8, 0.8, 0.8}}c|}{\textbf{ACCI�N}} & \multicolumn{1}{|>{\columncolor[rgb]{0.8, 0.8, 0.8}}c|}{\textbf{ROL}} \\
\hline
Registrar beneficiario & Administrador, Citador\\
\hline
Consultar beneficiario & Administrador, Citador, M�dico\\
\hline
Modificar beneficiario & Administrador, Citador\\
\hline
Eliminar beneficiario & Administrador, Citador\\
\hline
\colorFila Consultar beneficiarios asignados a un m�dico & Administrador\\
\hline
Registrar usuario & Administrador\\
\hline
Consultar usuario & Administrador\\
\hline
Modificar usuario & Administrador\\
\hline
Eliminar usuario & Administrador\\
\hline
\colorFila Corresponde NIF a un usuario & Administrador, Citador, M�dico\\
\hline
\colorFila Consultar usuario que ha iniciado sesi�n & Administrador, Citador, M�dico\\
\hline
\colorFila Consultar centros de salud & Administrador, Citador, M�dico\\
\hline
\colorFila Registrar m�dico & Administrador\\
\hline
\colorFila Consultar m�dico & Administrador, Citador\\
\hline
\colorFila Modificar m�dico & Administrador\\
\hline
\colorFila Eliminar m�dico & Administrador\\
\hline
\colorFila Consultar m�dicos de un tipo determinado & Administrador\\
\hline
\colorFila Consultar posibles sustitutos & Administrador\\
\hline
Establecer sustituto & Administrador\\
\hline
\colorFila Consultar m�dico que atiende una cita & Administrador, Cita\\
\hline
Tramitar cita & Administrador, Citador\\
\hline
Tramitar cita a partir de un volante & Administrador, Citador\\
\hline
\colorFila Consultar horario de un m�dico & Administrador, Citador\\
\hline
\colorFila Consultar horas de citas asignadas a un m�dico & Administrador, Citador\\
\hline
\colorFila Consultar d�as completos de un m�dico & Administrador, Citador\\
\hline
Consultar citas de un beneficiario & Administrador, Citador\\
\hline
Consultar citas de un m�dico & Administrador, Citador\\
\hline
Anular cita & Administrador, Citador\\
\hline
Consultar citas propias de un m�dico & M�dico\\
\hline
\colorFila Consultar citas en una fecha de un m�dico & Administrador, Citador\\
\hline
Emitir Volante & M�dico\\
\hline
\colorFila Consultar Volante & Administrador, Citador, M�dico\\
\hline
\caption{Operaciones soportadas en el sistema}
\label{tb:operaciones}
\end{longtable}

Estas operaciones auxiliares se detallan en la siguiente secci�n, donde se comentar� el uso principal de cada una de ellas y a cu�l de las acciones que puede realizar el usuario (es decir, a qu� funcionalidad del sistema, ver tabla \ref{tb:operacionesInterfaz}) dan soporte. 

----------------- COMPLETAR ------------------