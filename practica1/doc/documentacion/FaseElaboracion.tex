%
% Al que le toque hacer esta parte, que ponga esto cuando hable de los
% diagramas de clases de an�lisis; lo iba a poner en decisiones de dise�o,
% pero los gestores aparecen desde la fase del an�lisis
% (Juan G.)
%

Los dos gestores que aparecen en la mayor�a de los diagramas de clases de
an�lisis son una parte importante del sistema, cuyo funcionamiento ser� el
siguiente:
\begin{milista}
\item \textbf{Gestor de conexiones de bases de datos}. Servir� para mantener
sincronizadas las bases de datos utilizadas por el sistema, de manera que
cuando haya que hacer una inserci�n/modificaci�n/eliminaci�n, se haga en todas
las bases de datos, y si una operaci�n falla se siga manteniendo la
sincronizaci�n, lo cual ser� imprescindible para el correcto funcionamiento
del sistema. En este caso, el gestor acceder� a dos clases:
	\subitem \textit{AgenteFrontend}, que es la interfaz de acceso a la base de
	datos principal.
	\subitem \textit{ProxyServidorRespaldo}, que se comunica con el servidor de
	respaldo para poder utilizar la base de datos secundaria.
\item \textbf{Gestor de conexiones de estado}. Servir� para que los mensajes
de estado generados por el servidor front-end (principalmente despu�s de
que un usuario realice una operaci�n) puedan ser procesados por todas las
clases que los quieran recibir. En este caso, cada mensaje ser� utilizado por
tres clases:
	\subitem \textit{GestorConexionesBD}, para guardar el mensaje en la base
	de datos y as� mantener un registro de todas las operaciones realizadas,
	como se pide en los requisitos.
	\subitem \textit{JFServidorFrontend}, para mostrar el mensaje en la ventana
	principal del servidor front-end.
	\subitem \textit{ProxyServidorRespaldo}, para enviar el mensaje al servidor
	de respaldo y que �ste lo muestre en su ventana	principal.
\end{milista}
