\section{Fase de Elaboraci�n}

\subsection{Arquitectura cliente-servidor} \label{ClienteServidor}
Una vez detallados todos los requisitos funcionales que debe cumplir el sistema de citaci�n de una comunidad aut�noma, se procede a la divisi�n del sistema en los tres subsistemas que se plantean en la especificaci�n inicial de requisitos (ver secci�n \ref{requisitosIniciales}). Estos subsistemas son: cliente o unidad de citaci�n, servidor front-end y servidor de respaldo.

Adem�s, el sistema seguir� una arquitectura cliente-servidor, donde cada subsistema se comunica con otro u otros subsistemas, de la siguiente forma:

\begin{milista}
	\item \textbf{Subsistema Cliente:} este subsistema muestra una interfaz gr�fica a los usuarios del sistema de citaci�n, enviando la acci�n realizada por el usuario al servidor front-end, esperando su respuesta para actualizar la interfaz gr�fica en consecuencia. 
	\item \textbf{Subsistema Servidor front-end:} este subsistema se encarga de recibir las peticiones del subsistema cliente, procesarlas, almacenar y recuperar informaci�n de su base de datos y enviar la respuesta al cliente. Tambi�n se encarga de enviar la petici�n al servidor de respaldo, para que todas las acciones queden registradas tanto en la base de datos principal como en la de respaldo.
	\item \textbf{Subsistema Servidor de respaldo:} se encarga de recibir las peticiones del servidor front-end y ejecutarlas sobre las base de datos de respaldo. 
\end{milista}

As�, cada uno de los subsistemas se puede encontrar en m�quinas diferentes, pues, siguiendo la arquitectura cliente-servidor, se comunican a trav�s de RMI (Remote Method Invocation). Adem�s, la base de datos principal y la base de datos de respaldo tambi�n se pueden encontrar en m�quinas diferentes a las m�quinas donde se encuentren los servidores. 
Por tanto, algunas ventajas de utilizar este enfoque distribuido siguiendo la arquitectura cliente-servidor son: 

\begin{milista}
    \item \textbf{Centralizaci�n del control:} toda la l�gica de dominio y control est� centralizada en el servidor front-end, por lo que el sistema cliente es totalmente independiente de la implementaci�n del servidor. Del mismo modo, el sistema cliente es independiente de como se realiza la gesti�n del conocimiento en el servidor. Adem�s, los accesos, recursos y la integridad de los datos son controlados por el servidor de forma que un cliente defectuoso o no autorizado no pueda da�ar el sistema.
    \item \textbf{Escalabilidad:} se pueden a�adir nuevos tipos de sistemas clientes para que se comuniquen con el servidor.
    \item \textbf{F�cil mantenimiento:} al estar los sistemas distribuidos en diferentes m�quinas, es posible reemplazar, reparar o actualizar un servidor, mientras que sus clientes no se ver�n afectados por ese cambio.
\end{milista}

En la Figura \ref{fig:clienteServidor} se muestra una vista de los subsistemas que integran el sistema de salud y como se comunican dichos sistemas a trav�s de interfaces, facilitando la distribuci�n de cada uno de los sistemas en diferentes m�quinas. Este diagrama se puede consultar en el \diagrama{sistema}{despliegue}{Despliegue}.

\imagen{./imagenes/Despliegue}{0.35}{Vista arquitectura cliente-servidor}{fig:clienteServidor}

\subsection{Diagrama de Casos de Uso}
En los siguientes apartados se van a comentar los casos de uso que se han identificado en cada uno de los tres subsistemas comentados en el punto anterior. Dichos casos de uso van a representar las funcionalidades que debe tener el sistema de citaci�n y que ya se han comentado en los apartados \ref{requisitosIniciales} y \ref{requisitos-detallados}.

\subsubsection{Unidad de citaci�n}

Los casos de uso del subsistema del cliente tienen un mecanismo com�n de funcionamiento, como ya se vi� en la arquitectura cliente-servidor (ver secci�n \ref{ClienteServidor}): se solicitan al usuario los datos que correspondan a la funcionalidad de ese caso de uso (datos del beneficiario, datos del m�dico, etc.) y se envian dichos datos al servidor, esperando su respuesta y mostrando los resultados. Por esta raz�n, los actores que aparecen en el diagrama de casos de uso de este subsistema son los diferentes usuarios del sistema y el servidor front-end.

Por otra parte, seg�n la especifiaci�n de requisitos, se han extraido los siguientes casos de uso:

\begin{milista}
	\item \textbf{Iniciar Sesi�n:} este caso de uso, que puede ser ejecutado por cualquier usuario del sistema, es el que representa la identificaci�n de un usuario en el sistema y, por tanto, su inicio de sesi�n en �l. Es el primer caso de uso que debe ejecutarse en el sistema de citaci�n, por lo que todos los dem�s casos de uso incluyen el comportamiento de �ste. Pero, para facilitar la legibilidad del diagrama, no se han representado todas las relaciones de inclusi�n ("`include"') del resto de casos de uso hacia este caso.
	\item \textbf{Cerrar Sesi�n:} este caso de uso, que puede ser ejecutado por cualquier usuario del sistema, es el que representa la desconexi�n de un usuario del sistema de citaci�n.
	\item \textbf{Registrar Beneficiario:} este caso de uso, que puede ser ejecutado por el administrador o el citador del sistema, representa la funcionalidad para registrar los datos de un beneficiario del sistema.
	\item \textbf{Consultar Beneficiario:} este caso de uso, que puede ser ejecutado por cualquier usuario del sistema, representa la funcionalidad para consultar los datos de un beneficiario registrado en el sistema.
	\item \textbf{Modificar Beneficiario:} este caso de uso, que puede ser ejecutado por el administrador o el citador del sistema, representa la funcionalidad para modificar los datos de un beneficiario registrado en el sistema. Para ello, se incluye la funcionalidad del caso de uso anterior, pues primero se consultan todos los datos del beneficiario y desp�es se modifican aquellos datos que el usuario desee.
	\item \textbf{Eliminar Beneficiario:} este caso de uso, que puede ser ejecutado por el administrador o el citador del sistema, representa la funcionalidad para eliminar un beneficiario registrado en el sistema.
		\item \textbf{Registrar Usuario:} este caso de uso, que puede ser ejecutado por el administrador del sistema, representa la funcionalidad para registrar un nuevo usuario (administrador, citador o m�dico) del sistema. En el caso de registrar un nuevo m�dico, se debe introducir tambi�n los datos del calendario de dicho m�dico.
	\item \textbf{Consultar Usuario:} este caso de uso, que puede ser ejecutado por el administrador del sistema, representa la funcionalidad para consultar los datos de un usuario registrado en el sistema. En el caso de consultar los datos de un m�dico, tambi�n se consultar� su calendario de trabajo. 
	\item \textbf{Modificar Usuario:} este caso de uso, que puede ser ejecutado por el administrador del sistema, representa la funcionalidad para modificar los datos de un usuario registrado en el sistema. Para ello, se incluye la funcionalidad del caso de uso anterior, pues primero se consultan todos los datos del usuario y desp�es se modifican aquellos datos que se deseen. En el supuesto de modificar un m�dico, tambi�n se podr�a modificar su calendario laboral. 
	\item \textbf{Eliminar Usuario:} este caso de uso, que puede ser ejecutado por el administrador del sistema, representa la funcionalidad para eliminar un usuario registrado en el sistema.
	\item \textbf{Tramitar Cita:} este caso de uso, que puede ser ejecutado por el administrador o el citador del sistema, representa la funcionalidad para registrar una cita de un beneficiario con su m�dico. En el supuesto de que el beneficiario que desea pedir la cita no exista en el sistema, se podr� registrar. Adem�s, se puede tramitar una cita para un volante, siempre que el beneficiario posea un volante para un especialista.
	\item \textbf{Consultar citas:} este caso de uso, que puede ser ejecutado por el administrador o el citador del sistema, representa la funcionalidad para poder consultar las citas de un beneficiario o de un m�dico.
		\item \textbf{Anular cita:} este caso de uso, que puede ser ejecutado por el administrador o el citador del sistema, representa la funcionalidad para poder anular una cita de un beneficiario. Se incluye la funcionalidad del caso de uso \textit{Consultar citas beneficiario}, pues primero se consultan todas las citas de ese beneficiario y despu�s se anula la cita que dicho beneficiario desee.
	\item \textbf{Emitir Volante:} este caso de uso reperesenta la funcionalidad de emisi�n de volantes para un beneficiario por parte de un m�dico. 
	\item \textbf{Establecer Sustituto:} este caso de uso, que puede ser ejecutado por el administrador del sistema, representa la funcionalidad para asignar un m�dico sustituto a otro m�dico.
	\item \textbf{Actualizar Ventanas:} este caso de uso reperesenta la funcionalidad para actualizar la interfaz gr�fica del subsistema cliente cuando se recibe una notificaci�n desde el servidor. La interfaz gr�fica se actualiza si la operaci�n que se ha notificado desde el servidor afecta a la ventana que se est� mostrando al usuario. 
\end{milista}

Siguiendo con la especificaci�n de requisitos detallada del apartado \ref{requisitos-detallados}, los casos de uso que pertenecen a cada uno de los gestores definidos en dicha especificaci�n son: 

\begin{milista}
	\item \textbf{Gesti�n de beneficiarios}
	\begin{itemize}
		\item Registrar beneficiario
		\item Consultar beneficiario
		\item Modificar beneficiario
		\item Eliminar beneficiario
	\end{itemize}
	\item \textbf{Gesti�n de usuarios}
	\begin{itemize}
		\item Registrar usuario
		\item Consultar usuario
		\item Modificar usuario
		\item Eliminar usuario
	\end{itemize}
	\item \textbf{Gesti�n de citas}
	\begin{itemize}
		\item Tramitar citas
		\item Consultar citas
		\item Anular cita
	\end{itemize}
	\item \textbf{Gesti�n de volantes}
	\begin{itemize}
		\item Emitir volante
	\end{itemize}
	\item \textbf{Gesti�n de sustituciones}
	\begin{itemize}
		\item Establecer sustituto
	\end{itemize}
\end{milista}

Para terminar, se ha utilizado una jerarqu�a de herencia entre los diferentes tipos de usuarios, pus hay operaciones que pueden realizar varios de ellos (por ejemplo, el administrador puede realizar las mismas operaciones que un citador, m�s las operaciones propias de adminsitrador).

Este diagrama de casos de uso puede consultarse en el \diagrama{cliente}{casosUso}{Casos de Uso}.

\subsubsection{Servidor front-end}

En este subsistema, como ya se vi� en la arquitectura cliente-servidor (ver secci�n \ref{ClienteServidor}), el modo de proceder es el siguiente: se reciben las peticiones del subsistema cliente, se opera con la l�gica de control y de dominio, accediendo a la base de datos, se env�a la operaci�n al servidor de respaldo y a la base de datos principal y se env�a la respuesta al cliente, notificando los posibles cambios. Por esta raz�n, los actores que aparecen en el diagrama de casos de uso de este subsistema son el subsistema cliente, el SGDB y el servidor de respaldo. 

En cuanto a los casos de uso, aparecen los mismos que en el subsistema anterior, salvo los que se comentan a continuaci�n:

\begin{milista}
	\item \textbf{Enviar Petici�n:} este caso de uso representa la funcionalidad para enviar una operaci�n al servidor de respaldo, para que se ejecute en la base de datos de respaldo.
	\item \textbf{Actualizar Estado:} ............
\end{milista}

Este diagrama de casos de uso puede consultarse en el \diagrama{frontend}{casosUso}{Casos de Uso}.

\subsubsection{Servidor de respaldo}

En este subsistema, los actores que aparecen en el diagrama de casos de uso son el subsistema del servidor front-end, del que recibe las peticiones, y el SGDB de la base de datos de respaldo.

En cuanto a los casos de uso, en el servidor de respaldo s�lo se definen dos casos:

\begin{milista}
	\item \textbf{Ejecutar Petici�n:} este caso de uso representa la funcionalidad para recibir una operaci�n del servidor front-end, ejecutarla en la base de datos de respaldo y devolver una respuesta al servidor front-end.
	\item \textbf{Actualizar Estado:} ............
\end{milista}

Este diagrama de casos de uso puede consultarse en el \diagrama{respaldo}{casosUso}{Casos de Uso}.



\subsection{Diagrama de Clases de An�lisis}

%
% Al que le toque hacer esta parte, que ponga esto cuando hable de los
% diagramas de clases de an�lisis; lo iba a poner en decisiones de dise�o,
% pero los gestores aparecen desde la fase del an�lisis
% (Juan G.)
%
Los dos gestores que aparecen en la mayor�a de los diagramas de clases de
an�lisis son una parte importante del sistema, cuyo funcionamiento ser� el
siguiente:
\begin{milista}
\item \textbf{Gestor de conexiones de bases de datos}. Servir� para mantener
sincronizadas las bases de datos utilizadas por el sistema, de manera que
cuando haya que hacer una inserci�n/modificaci�n/eliminaci�n, se haga en todas
las bases de datos, y si una operaci�n falla se siga manteniendo la
sincronizaci�n, lo cual ser� imprescindible para el correcto funcionamiento
del sistema. En este caso, el gestor acceder� a dos clases:
	\subitem \textit{AgenteFrontend}, que es la interfaz de acceso a la base de
	datos principal.
	\subitem \textit{ProxyServidorRespaldo}, que se comunica con el servidor de
	respaldo para poder utilizar la base de datos secundaria.
\item \textbf{Gestor de conexiones de estado}. Servir� para que los mensajes
de estado generados por el servidor front-end (principalmente despu�s de
que un usuario realice una operaci�n) puedan ser procesados por todas las
clases que los quieran recibir. En este caso, cada mensaje ser� utilizado por
tres clases:
	\subitem \textit{GestorConexionesBD}, para guardar el mensaje en la base
	de datos y as� mantener un registro de todas las operaciones realizadas,
	como se pide en los requisitos.
	\subitem \textit{JFServidorFrontend}, para mostrar el mensaje en la ventana
	principal del servidor front-end.
	\subitem \textit{ProxyServidorRespaldo}, para enviar el mensaje al servidor
	de respaldo y que �ste lo muestre en su ventana	principal.
\end{milista}


\subsection{Modelo de dominio} % Vista conceptual
Una vez elaborados los diagramas de casos de uso y los diagramas de clases de an�lisis, se procede a elaborar un modelo de dominio o modelo de conocimiento.

En dicho modelo se representan las clases y relaciones que pueden extraerse del dominio del sistema, es decir, se modelan las clases y relaciones que aparecen en la especificaci�n de requisitos de las secciones \ref{requisitosIniciales} y \ref{requisitos-detallados}.

As�, seg�n esta especificaci�n de requisitos, se han modelado las siguientes clases de conocimiento: 

\begin{milista}
	\item 
\end{milista}


