\paragraph{Capa de persistencia} \label{persistencia-front-end}

En esta capa, entre otras decisiones de dise�o que se comentar�n a continuaci�n, se hace uso de los patrones fabricaci�n pura, comando, agente y singleton. Para una mejor comprensi�n se recomienda ver el \diagrama{cliente}{clases}{Gestor Conexiones BBDD}.

Para abstraer al programador de las caracter�sticas espec�ficas del lenguaje de programaci�n empleado (Java) al realizar operaciones en la base de datos se utiliza el patr�n \textbf{Comando} o \textbf{Command}, permiti�ndo encapsular en una clase de m�s alto nivel la construcci�n de la sentencia que se ejecutar� en la base de datos. La clase \textit{ComandoSQL} tiene dos especializaciones: \textit{ComandoSQLProcedimiento} y \textit{ComandoSQLSentencia}. De este modo, el programador �nicamente deber� pasar la sentencia SQL y la lista de par�metros de n�mero variable que hay que incrustar en ella, abstray�ndole por completo de la creaci�n de sentencias espec�ficas del lenguaje (\textit{PreparedStatement} y \textit{CallabeStatement}).

Para proveer de un acceso al SGBD (Sistema Gestor de Bases de Datos) se ha implementado una clase \textbf{Agente} o \textbf{Broker} junto con el patr�n \textbf{Singleton}. El agente es el que inicializa, mantiene y cierra la conexi�n con el SGBD, manteni�ndose una sola instancia de �ste debido a que es singleton. De este modo se evita se evita la creaci�n de m�ltiples instancias del agente para no saturar el SGBD. Adem�s, se encarga de ejecutar las sentencias encapsuladas en el ComandoSQL. Por otro lado, n�tese que implementa las acciones \textit{commit()} y \textit{rollback()}.

En referencia a la persistencia de las clases de conocimiento persistentes, se han tenido en cuenta dos alternativas: \textbf{patr�n experto} y \textbf{patr�n de fabricaci�n pura}. Para ser fieles con el \underline{principio de m�xima coherencia y bajo acomplamiento}, se ha descartado el patr�n experto ya que se acoplar�an las clases de conocimiento con la persistencia, violando as� dicho principio. Por tanto se ha utilizando el patr�n de fabricaci�n pura implementando clases que tienen como �nica responsabilidad las operaciones CRUD (Create, Read, Update, Delete) de las clases persistentes, de tal modo que si se cambia el SGDB s�lo se tendr�an que cambiar el modo en que dichas clases definen las sentencias SQL, ya que la creaci�n est� delegada en el ComandoSQL. Adem�s, cada una de estas clases de fabricaci�n pura se encargan de acceder exclusivamente a una tabla de la base de datos, reduciendo as� al m�ximo el acoplamiento entre clases de persistencia.