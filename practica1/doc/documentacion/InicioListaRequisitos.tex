\subsection{Lista de requisitos funcionales}

Despu�s de analizar con detalle los requisitos y las consideraciones tenidas
en cuenta para el desarrollo del sistema, en este apartado se listan todos los
requisitos funcionales del sistema.

\subsubsection{Unidad de citaci�n}

\begin{milista}
\item \textbf{RF1}: Identificar a los usuarios del sistema.
\item \textbf{RF2}: Consultar los datos de un beneficiario.
\item \textbf{RF3}: Registrar un nuevo beneficiario.
\item \textbf{RF4}: Modificar los datos de un beneficiario.
\item \textbf{RF5}: Eliminar un beneficiario.
\item \textbf{RF6}: Consultar los datos de un usuario (o m�dico).
\item \textbf{RF7}: Registrar un nuevo usuario (o m�dico).
\item \textbf{RF8}: Modificar los datos de un usuario (o m�dico).
\item \textbf{RF9}: Eliminar un usuario (o m�dico).
\item \textbf{RF10}: Asignar sustitutos a los m�dicos.
\item \textbf{RF11}: Consultar las citas de un beneficiario.
\item \textbf{RF12}: Consultar las citas de un m�dico.
\item \textbf{RF13}: Registrar una nueva cita.
\item \textbf{RF14}: Anular una cita.
\item \textbf{RF15}: Emitir un volante.
\item \textbf{RF16}: Recibir notificaciones para actualizar el estado.
\end{milista}

\subsubsection{Servidor front-end}

\begin{milista}
\item \textbf{RF1}: Identificar y registrar a los clientes.
\item \textbf{RF2}: Obtener los datos de un beneficiario.
\item \textbf{RF3}: Almacenar un nuevo beneficiario.
\item \textbf{RF4}: Modificar los datos de un beneficiario.
\item \textbf{RF5}: Eliminar un beneficiario.
\item \textbf{RF6}: Obtener los datos de un usuario (o m�dico).
\item \textbf{RF7}: Almacenar un nuevo usuario (o m�dico).
\item \textbf{RF8}: Modificar los datos de un usuario (o m�dico).
\item \textbf{RF9}: Eliminar un usuario (o m�dico).
\item \textbf{RF10}: Almacenar sustituciones a m�dicos.
\item \textbf{RF11}: Consultar las citas de un beneficiario.
\item \textbf{RF12}: Consultar las citas de un m�dico.
\item \textbf{RF13}: Almacenar una nueva cita.
\item \textbf{RF14}: Eliminar una cita.
\item \textbf{RF15}: Almacenar un nuevo volante.
\item \textbf{RF16}: Registrar todas las operaciones realizadas por los
usuarios en la BD.
\item \textbf{RF17}: Actualizar el estado de los clientes cuando se produce una
operaci�n en el servidor.
\item \textbf{RF18}: Enviar peticiones al servidor de respaldo para gestionar
la base de datos secundaria.
\end{milista}

\subsubsection{Servidor de respaldo}

\begin{milista}
\item \textbf{RF1}: Recibir peticiones para gestionar la base de datos secundaria.
\end{milista}

