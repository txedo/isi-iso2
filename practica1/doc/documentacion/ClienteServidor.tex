Una vez detallados todos los requisitos funcionales que debe cumplir el sistema de citaci�n de una comunidad aut�noma, se procede a la divisi�n del sistema en los tres subsistemas que se plantean en la especificaci�n inicial de requisitos (ver secci�n \ref{requisitosIniciales}). Estos subsistemas son: cliente o unidad de citaci�n, servidor front-end y servidor de respaldo.

Adem�s, el sistema seguir� una arquitectura cliente-servidor, donde cada subsistema se comunica con otro u otros subsistemas, de la siguiente forma:

\begin{milista}
	\item \textbf{Subsistema Cliente:} este subsistema muestra una interfaz gr�fica a los usuarios del sistema de citaci�n, enviando la acci�n realizada por el usuario al servidor front-end, esperando su respuesta para actualizar la interfaz gr�fica en consecuencia. 
	\item \textbf{Subsistema Servidor front-end:} este subsistema se encarga de recibir las peticiones del subsistema cliente, procesarlas, almacenar y recuperar informaci�n de su base de datos y enviar la respuesta al cliente. Tambi�n se encarga de enviar la petici�n al servidor de respaldo, para que todas las acciones queden registradas tanto en la base de datos principal como en la de respaldo.
	\item \textbf{Subsistema Servidor de respaldo:} se encarga de recibir las peticiones del servidor front-end y ejecutarlas sobre las base de datos de respaldo. 
\end{milista}

As�, cada uno de los subsistemas se puede encontrar en m�quinas diferentes, pues, siguiendo la arquitectura cliente-servidor, se comunican a trav�s de RMI (Remote Method Invocation). Adem�s, la base de datos principal y la base de datos de respaldo tambi�n se pueden encontrar en m�quinas diferentes a las m�quinas donde se encuentren los servidores. 
Por tanto, algunas ventajas de utilizar este enfoque distribuido siguiendo la arquitectura cliente-servidor son: 

\begin{milista}
    \item \textbf{Centralizaci�n del control:} toda la l�gica de dominio y control est� centralizada en el servidor front-end, por lo que el sistema cliente es totalmente independiente de la implementaci�n del servidor. Del mismo modo, el sistema cliente es independiente de como se realiza la gesti�n del conocimiento en el servidor. Adem�s, los accesos, recursos y la integridad de los datos son controlados por el servidor de forma que un cliente defectuoso o no autorizado no pueda da�ar el sistema.
    \item \textbf{Escalabilidad:} se pueden a�adir nuevos tipos de sistemas clientes para que se comuniquen con el servidor.
    \item \textbf{F�cil mantenimiento:} al estar los sistemas distribuidos en diferentes m�quinas, es posible reemplazar, reparar o actualizar un servidor, mientras que sus clientes no se ver�n afectados por ese cambio.
\end{milista}

En la Figura \ref{fig:clienteServidor} se muestra una vista de los subsistemas que integran el sistema de salud y como se comunican dichos sistemas a trav�s de interfaces, facilitando la distribuci�n de cada uno de los sistemas en diferentes m�quinas. Este diagrama se puede consultar en el \diagrama{sistema}{despliegue}{Despliegue}.

\imagen{./imagenes/Despliegue}{0.35}{Vista arquitectura cliente-servidor}{fig:clienteServidor}