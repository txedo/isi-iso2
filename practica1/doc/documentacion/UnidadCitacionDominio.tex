\paragraph{Capa de dominio}

En el sistema cliente, la capa de dominio s�lo contiene clases relacionadas con el control de este sistema, careciendo de clases de conocimiento, ya que todas esas clases se encuentran en el lado del servidor. De este modo, el cliente se limita a realizar peticiones al servidor (utilizando la capa de comunicaciones), obtener su respuesta y gestionar la capa de presentaci�n seg�n sea necesario. 

Una de las clases de este paquete de control, llamada \textit{ControladorCliente}, se centra en controlar los diferentes elementos de la capa de presentaci�n y de comunicarse con el servidor front-end, utilizando la capa de comunicaciones.

Por otra parte, la clase \textit{Cliente} es la clase que implementa la interfaz \textit{ICliente}, que, entre otros m�todos, tiene un m�todo para que el controlador actualice la interfaz gr�fica con las respuestas que obtiene del servidor, cuando otro cliente hace una petici�n. 

Estas comunicaciones entre capas se pueden observar en el \diagrama{cliente}{paquetes}{Divisi�n en Capas}.

