\section{Aplicaci�n de ejemplo}

Para estudiar m�s a fondo las caracter�sticas de los kits y entornos de desarrollo disponibles para Symbian OS se ha programado una peque�a aplicaci�n de ejemplo, adaptando para ello un sencillo juego 3D realizado para la asignatura de Inform�tica Gr�fica de la UCLM.

La aplicaci�n ha sido desarrollada en Symbian C++ utilizando el SDK S60 3rd Edition, por ser el que ofrece m�s compatibilidad con los tel�fonos m�viles que hay actualmente en el mercado. Las herramientas utilizadas para programar, compilar y probar la aplicaci�n son las siguientes:
\begin{milista}
	\item Carbide.c++ 2.3. \footnote{\url{http://www.forum.nokia.com/info/sw.nokia.com/id/dbb8841d-832c-43a6-be13-f78119a2b4cb.html}}
	\item S60 3rd Edition SDK for Symbian OS, Feature Pack 2. \footnote{\url{http://www.forum.nokia.com/info/sw.nokia.com/id/ec866fab-4b76-49f6-b5a5-af0631419e9c/S60_All_in_One_SDKs.html}}
	\item OpenGL ES 1.1 plug-in for S60 3rd Edition SDK. \footnote{\url{http://www.forum.nokia.com/info/sw.nokia.com/id/36331d44-414a-4b82-8b20-85f1183e7029/OpenGL_ES_1_1_Plug_in.html}}
	\item ActivePerl 5.10.1.1007. \footnote{\url{http://www.activestate.com/activeperl/downloads/}}
\end{milista}

A continuaci�n se explicar� la estructura de directorios de la aplicaci�n de ejemplo que se ha desarrollado, que es similar a la 

\begin{milista}
	\item \textbf{data}: 
	\item \textbf{gfx}: 
	\item \textbf{group}: 
	\item \textbf{src}: 
	\item \textbf{inc}: contiene los ficheros de cabecera \em{.h} correspondientes a los ficheros de c�digo fuente \em{.cpp} del directorio \emph{src}.
	\item \textbf{sis}: 
\end{milista}

%%%%%%%%%%%%%%%%%%%%%%%%%%%%%%%%%%%%%%poner pantallazo de la aplicaci�n
%%%%%%%%%%%%%%%%%%%%%%%%%%%%%%%%%%%%%%comentar que se ha probado en emulador y un dispostivo real
