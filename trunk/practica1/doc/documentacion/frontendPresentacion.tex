\paragraph{Capa de presentaci�n} \label{presentacion-front-end}

En el servidor front-end, la capa de presentaci�n no tiene demasiada
relevancia, ya que s�lo se encarga de mostrar la ventana de estado del servidor
al usuario y permitirle conectar, desconectar y configurar el servidor
front-end. Se tom� la decisi�n de mostrar los mensajes de estado en el servidor
(y no simplemente guardarlos en la base de datos) para que el administrador
encargado de mantener el servidor tenga realimentaci�n sobre las operaciones
que realiza el sistema mientras est� en ejecuci�n.

La clase encargada de mostrar los mensajes de estado en la ventana del
servidor front-end es \textit{ConexionLogVentana}, una clase del paquete de
comunicaciones a la que se hace referencia en la secci�n
\ref{comunicaciones-front-end}. En esta clase se ha aplicado el patr�n
\textbf{Observador} para permitir que los mensajes se puedan mostrar en m�s
de una ventana a la vez, sin necesidad de crear varias instancias de
\textit{ConexionLogVentana}.

As� pues, como se observa en el \diagrama{frontend}{clases}{Gestor Conexiones
Log}, la clase \textit{ConexionLogVentana} tiene asociado cero, uno o m�s
instancias de la interfaz \textit{IVentanaEstado}, a las que reenv�a todos los
mensajes que le llegan. En la implementaci�n final del sistema, s�lo hay una
clase que implementa esta interfaz, que es la ventana principal del servidor
front-end, \textit{JFServidorFrontend}, pero el uso del patr�n
\textbf{Observador} facilita notablemente la extensibilidad del sistema.
