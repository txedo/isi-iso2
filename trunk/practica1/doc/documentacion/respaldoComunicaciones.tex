\paragraph{Capa de comunicaciones}

En el paquete de comunicaciones del servidor de respaldo no se utiliza el
patr�n \textbf{Proxy} porque este sistema no se tiene que comunicar de forma
activa con ning�n otro sistema, sino que es el servidor front-end el que
utiliza los servicios del servidor de respaldo.

Como se puede observar en el \diagrama{respaldo}{clases}{Servidor de respaldo
remoto}, la interfaz \textit{IServidorRespaldo} que contiene las operaciones
que se pueden solicitar al servidor de respaldo es simplemente una agrupaci�n
de las interfaces \textit{IConexionBD} e \textit{IConexionLog}. Se decidi�
crear la interfaz de esta forma para que quedara claro que el servidor de
respaldo tiene dos funcionalidades bien diferenciadas: acceder a la base de
datos secundaria y mostrar los mensajes de estado en la ventana del servidor
de respaldo.

Una gran ventaja que se deriva de crear la interfaz \textit{IServidorRespaldo}
de esta forma es que s�lo es necesario exportar un objeto remoto para acceder
a toda la funcionalidad del servidor de respaldo. Inicialmente, se plante� la
posibilidad de exportar un objeto dedicado exclusivamente a acceder a la base
de datos y otro que fuera el encargado de mostrar los mensajes en la ventana
del servidor, pero eso ten�a el inconveniente de que se necesitaban dos puertos
diferentes para mantener la comunicaci�n, haciendo poco extensible el sistema
en caso de que posteriormente se a�adieran nuevas funcionalidades.

Adem�s, otra ventaja es que se puede utilizar el objeto remoto que implementa
la interfaz del servidor en los m�todos donde se necesita una
\textit{IConexionBD} y en los que hace falta una \textit{IConexionLog}.
Adem�s, si en el futuro se tuviera que a�adir una nueva funcionalidad al
servidor de respaldo, bastar�a con hacer que la interfaz
\textit{IServidorRespaldo} implementara otra interfaz m�s, no afectando al
c�digo existente.

Por otro lado, al igual que se coment� en el paquete de comunicaciones del
servidor front-end (ver secci�n \ref{comunicaciones-front-end}), tambi�n se
busca una IP p�blica o privada autom�ticamente al exportar el objeto remoto
\textit{RemotoServidorRespaldo}, para que funcione si el equipo donde se
ejecuta el servidor pertenece a m�s de una red.
