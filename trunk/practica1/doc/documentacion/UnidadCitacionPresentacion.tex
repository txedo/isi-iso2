\paragraph{Capa de presentaci�n}

De las tres aplicaciones que forman el sistema construido, la unidad de
citaci�n es aquella en la que la capa de presentaci�n es m�s importante,
debido a que es la que utilizar� el usuario final para manipular el sistema de
salud.

El desarrollo de la interfaz gr�fica del cliente se ha basado en la divisi�n
del sistema que se hizo en la gesti�n de requisitos (ver secci�n
\ref{requisitos-detallados}). De esta forma, a cada gestor identificado
(beneficiarios, usuarios, citas, volantes y sustituciones) se le ha asignado
una pesta�a en la interfaz gr�fica, dentro de la cual se agrupan los paneles
de todas las operaciones relacionadas. Por sencillez para el usuario, las
operaciones de consulta, modificaci�n y eliminaci�n se han unido en una �nica
operaci�n, por lo que para eliminar o modificar un usuario primero se debe
consultar.

Un detalle interesante que conviene destacar de las operaciones disponibles
para cada usuario es que �stas no est�n fijadas en el c�digo de la unidad de
citaci�n, sino que cuando un usuario inicia sesi�n, pide las operaciones
que puede realizar al servidor front-end y, en funci�n de la respuesta
recibida, se muestran unas u otras operaciones. La gran ventaja de esta forma
de determinar las operaciones para un usuario es que, si se crea un nuevo tipo
de usuario o cambian las operaciones asignadas a un tipo de usuario existente,
los cambios que habr�a que hacer en la unidad de citaci�n ser�an m�nimos.

A la hora de implementar la interfaz, se tom� la decisi�n de crear un panel
(una clase) por cada operaci�n que el usuario puede realizar con el sistema.
Por tanto, hay un panel dedicado a registrar beneficiarios, otro a
consultar/modificar/eliminar usuarios, otro diferente a tramitar citas, etc.
Esta divisi�n ha permitido reutilizar paneles en otros paneles, aprovechando
as� la potencia de la orientaci�n a objetos. Por ejemplo, en el panel de
tramitaci�n de citas, \textit{JPCitaTramitar}, se utiliza el panel de
consulta de beneficiarios, \textit{JPBeneficiarioConsultar}, para elegir el
beneficiario que quiere pedir cita. En el \diagrama{cliente}{clases}{Ventana
Principal} se puede ver con detalle la jerarqu�a de paneles usada en la
interfaz gr�fica, as� como los paneles que est�n formados de otros paneles.

Debido a la gran cantidad de clases que ten�a el paquete de presentaci�n, se
decidieron separar todas aquellas clases que no eran ventanas o paneles de la
interfaz gr�fica y moverlas al subpaquete \textit{presentacion.auxiliar}. As�
pues, en este paquete se encuentran implementaciones personalizadas de algunos
componentes Java, definiciones de eventos usados en algunos paneles y clases
de utilidades con m�todos que se reutilizan en varios paneles.

Una clase especialmente importante en este subpaquete es \textit{Validacion},
que es donde se encuentran todos los m�todos encargados de validar los campos
introducidos en el usuario. Como se puede ver en los diagramas de secuencia del
cliente (por ejemplo, en el \diagrama{cliente}{secuencia}{Registrar M�dico}),
siempre que el usuario quiere realizar una operaci�n con la unidad de
citaci�n, se llama a la clase \textit{Validacion} para comprobar que los datos
de la operaci�n tienen el formato correcto, con el fin de evitar enviar al
servidor peticiones que se sabe que son incorrectas.
