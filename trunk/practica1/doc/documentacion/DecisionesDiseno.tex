\subsection{Vista L�gica}

En los diferentes puntos de esta secci�n se van a detallar las decisiones de dise�o que se han tenido en cuenta a la hora de dise�ar y desarrollar los tres sistemas: unidad de citaci�n o cliente; servidor front-end y servidor de respaldo.

\subsubsection{Arquitectura multicapa}
Cabe destacar que todos los sistemas se han desarrollado siguiendo una arquitectura multicapa. De este modo, las capas con las que cuenta cada sistema son las siguientes: 

\begin{milista}
	\item \textbf{Unidad de citaci�n}: comunicaciones, dominio y presentaci�n.
	\item \textbf{Servidor front-end}: comunicaciones, dominio, persistencia y presentaci�n.
	\item \textbf{Servidor de respaldo}: comunicaciones, dominio, persistencia y presentaci�n.
\end{milista}

Utilizando este enfoque multicapa, se sigue el principio de m�nimo acoplamiento y m�xima cohesi�n, desacoplando los elementos de una capa de los de otra. De este modo, se facilita el mantenimiento y extensibilidad del sistema, pues cuando se realice el cambio en alg�n elemento de una capa, el resto de capas no se ver�n afectadas.

En los siguientes apartados se comentar�n las decisiones de dise�o que se han tenido en cuenta para cada una de las capas de cada sistema desarrollado.

\paragraph{Capa de comunicaciones}

Del paquete de comunicaciones del cliente, cabe decir que se utiliza el patr�n
\textbf{Proxy} para comunicarse con el servidor front-end y que se busca de
manera autom�tica una IP p�blica o privada para exportar el objeto remoto del
cliente, de la misma forma que se explic� en la capa de comunicaciones del
servidor frontend (ver secci�n \ref{comunicaciones-front-end}).

Una decisi�n importante que se tom� al desarrollar esta parte del sistema fue
mover la definici�n de la interfaz \textit{ICliente} de la unidad de citaci�n,
donde se pens� colocar inicialmente, al servidor front-end. De esta manera, lo
que se consigue es que el servidor front-end sea totalmente independiente de
los clientes que vayan a utilizarlo, por lo que debe ser la unidad de citaci�n
la que tenga que ajustarse a la interfaz \textit{ICliente} ofrecida por el
servidor, y no al rev�s, que no tiene demasiado sentido en una arquitectura
cliente-servidor.

Adem�s, al dise�ar la unidad de citaci�n, se ha tenido en cuenta que en una
misma m�quina puede haber varios clientes ejecut�ndose a la vez, algo que no
tiene sentido en los servidores. Por este motivo, el puerto en el que cada
cliente pone su objeto remoto \textit{ICliente} a la escucha no es fijo, sino
que se selecciona din�micamente, probando varios puertos si los que se intentan
utilizar ya est�n en uso en el equipo.

\paragraph{Capa de dominio}
En el sistema cliente, la capa de dominio s�lo contiene clases para el control de este sistema, que se encargan de mostrar los diferentes elementos de la capa de presentaci�n y de comunicarse con el servidor front-end, utilizando para ello la capa de comunicaciones. 

\paragraph{Capa de presentaci�n}

De las tres aplicaciones que forman el sistema construido, la unidad de
citaci�n es aquella en la que la capa de presentaci�n es m�s importante,
debido a que es la que utilizar� el usuario final para manipular el sistema de
salud.

El desarrollo de la interfaz gr�fica del cliente se ha basado en la divisi�n
del sistema que se hizo en la gesti�n de requisitos (ver secci�n
\ref{requisitos-detallados}). De esta forma, a cada gestor identificado
(beneficiarios, usuarios, citas, volantes y sustituciones) se le ha asignado
una pesta�a en la interfaz gr�fica, dentro de la cual se agrupan los paneles
de todas las operaciones relacionadas. Por sencillez para el usuario, las
operaciones de consulta, modificaci�n y eliminaci�n se han unido en una �nica
operaci�n, por lo que para eliminar o modificar un usuario primero se debe
consultar.

Un detalle interesante que conviene destacar de las operaciones disponibles
para cada usuario es que �stas no est�n fijadas en el c�digo de la unidad de
citaci�n, sino que cuando un usuario inicia sesi�n, pide las operaciones
que puede realizar al servidor front-end y, en funci�n de la respuesta
recibida, se muestran unas u otras operaciones. La gran ventaja de esta forma
de determinar las operaciones para un usuario es que, si se crea un nuevo tipo
de usuario o cambian las operaciones asignadas a un tipo de usuario existente,
los cambios que habr�a que hacer en la unidad de citaci�n ser�an m�nimos.

A la hora de implementar la interfaz, se tom� la decisi�n de crear un panel
(una clase) por cada operaci�n que el usuario puede realizar con el sistema.
Por tanto, hay un panel dedicado a registrar beneficiarios, otro a
consultar/modificar/eliminar usuarios, otro diferente a tramitar citas, etc.
Esta divisi�n ha permitido reutilizar paneles en otros paneles, aprovechando
as� la potencia de la orientaci�n a objetos. Por ejemplo, en el panel de
tramitaci�n de citas, \textit{JPCitaTramitar}, se utiliza el panel de
consulta de beneficiarios, \textit{JPBeneficiarioConsultar}, para elegir el
beneficiario que quiere pedir cita. En el \diagrama{cliente}{clases}{Ventana
Principal} se puede ver con detalle la jerarqu�a de paneles usada en la
interfaz gr�fica, as� como los paneles que est�n formados de otros paneles.

Debido a la gran cantidad de clases que ten�a el paquete de presentaci�n, se
decidieron separar todas aquellas clases que no eran ventanas o paneles de la
interfaz gr�fica y moverlas al subpaquete \textit{presentacion.auxiliar}. As�
pues, en este paquete se encuentran implementaciones personalizadas de algunos
componentes Java, definiciones de eventos usados en algunos paneles y clases
de utilidades con m�todos que se reutilizan en varios paneles.

Una clase especialmente importante en este subpaquete es \textit{Validacion},
que es donde se encuentran todos los m�todos encargados de validar los campos
introducidos en el usuario. Como se puede ver en los diagramas de secuencia del
cliente (por ejemplo, en el \diagrama{cliente}{secuencia}{Registrar M�dico}),
siempre que el usuario quiere realizar una operaci�n con la unidad de
citaci�n, se llama a la clase \textit{Validacion} para comprobar que los datos
de la operaci�n tienen el formato correcto, con el fin de evitar enviar al
servidor peticiones que se sabe que son incorrectas.


\paragraph{Capa de persistencia}
La unidad de citaci�n representa el cliente dentro de una arquitectura cliente-servidor, por lo que no debe realizar ninguna acci�n relacionada con la persistencia, sino que estas operaciones debe delegarlas al servidor. 
\paragraph{Capa de comunicaciones} \label{comunicaciones-front-end}

En el paquete de comunicaciones se encuentran todas las clases relacionadas
con la conexi�n entre el servidor front-end y la unidad de citaci�n o el
servidor de respaldo. Para facilitar la comprensi�n y el mantenimiento tanto
de la unidad de citaci�n como de los dos servidores, se ha utilizado una
nomenclatura est�ndar para las clases principales de este paquete, de tal
forma que las clases que comienzan por \textit{Remoto...} sirven para exportar
objetos por RMI (como \textit{RemotoServidorFrontend}) y las clases que
comienzan por \textit{Proxy...} se emplean para comunicarse con objetos
remotos de otros sistemas (como \textit{ProxyCliente}).

Como se puede observar, para comunicarse con el cliente y el servidor de
respaldo, se ha hecho uso del patr�n \textbf{Proxy}. Para ello, en este
sistema se han creado dos clases \textit{ProxyCliente} y
\textit{ProxyServidorRespaldo} que son las responsables de establecer conexi�n
con la clases remotas exportadas por cada uno de los clientes registrados en
el sistema y por el servidor de respaldo.

Para comentar c�mo funciona el patr�n \textbf{Proxy} tomaremos como base la
comunicaci�n con el servidor de respaldo, siendo an�loga la comunicaci�n con
los clientes. En este caso, la clase \textit{ProxyServidorRespaldo} implementa
la interfaz \textit{IServidorRespaldo}, la cual contiene todas las operaciones
que se le pueden pedir al servidor de respaldo. De esta forma, el controlador
principal llama a la clase proxy de la misma forma que har�a si el servidor
de respaldo fuera local y estuviera implementado como parte del servidor
front-end pero, siguiendo el patr�n \textbf{Proxy}, la clase
\textit{ProxyServidorRespaldo} reenv�a todas las peticiones al objeto remoto
que est� en la m�quina del servidor de respaldo.

En este apartado conviene destacar que, con el fin de que el sistema funcione
correctamente incluso si la unidad de citaci�n o alguno de los servidores est�
dentro de varias redes, al exportar los objetos remotos se recorren todas las
interfaces de red para buscar una IP seg�n el siguiente orden:
\begin{enumerate}
	\item Si el ordenador pertenece a una red p�blica, se usa una IP p�blica.
	\item Si el ordenador no pertenece a una red p�blica pero s� a una privada,
	se utiliza una IP privada.
	\item Si el ordenador no est� conectado a ninguna red, se emplea la IP
	localhost (127.0.0.1).
\end{enumerate}

Adem�s, para que la comunicaci�n con los objetos remotos se establezca
correctamente, no s�lo es necesario indicar la IP al exportar los objetos, sino
que tambi�n hace falta modificar la propiedad
\texttt{java.rmi.server.hostname} de la m�quina virtual de Java (en los tres
sistemas), que representa la IP del servidor RMI que contiene los objetos.
Esta modificaci�n, sin embargo, se realiza en el controlador principal de cada
sistema, ya que no es responsabilidad de ning�n objeto remoto en particular.


\paragraph{Capa de dominio}

\paragraph{Capa de presentaci�n}



\paragraph{Capa de persistencia}

\subsubsection{Servidor de respaldo}

\paragraph{Capa de comunicaciones}
\paragraph{Capa de dominio}
\paragraph{Capa de presentaci�n}

Todas las decisiones que se tomaron para crear la capa de presentaci�n del
servidor front-end (ver secci�n \ref{presentacion-front-end}) son aplicables
para el servidor de respaldo, as� que no las comentaremos de nuevo en este
apartado.

\paragraph{Capa de persistencia}

Dado que el servidor de respaldo se limita a construir una r�plica exacta de la base de datos del SGBD principal en la base de datos del SGBD de respaldo, este sistema carece de cualquier clase de conocimiento. Debido a esto, tampoco contiene las clase de persistencia que se comentaban en la capa de persistencia del servidor front-end (ver \ref{persistencia-front-end}), ya que su �nica funci�n es ejecutar las sentencias que le env�a el servidor front-end.

Dicho esto, n�tese que la capa de persistencia s�lo contiene la clase que gestiona la conexiones al SGDB. Esta clase, al igual que en el servidor front-end (ver \ref{persistencia-front-end}), se ha dise�ado por medio de los patrones \textbf{Agente} y \textbf{Singleton}.







