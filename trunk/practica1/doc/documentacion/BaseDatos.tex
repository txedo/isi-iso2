\subsubsection{Desarrollo de la base de datos del sistema}

\paragraph{Dise�o conceptual}

El dise�o conceptual de la base de datos del sistema est� basado en las clases persistentes de la l�gica del dominio (ver \diagrama{frontend}{clases}{Modelo de Dominio}), representando dichas clases y sus relaciones en un model ER (Entidad-Interrelaci�n).
 
Como dicho modelo es el modelo conceptual (de m�s alto nivel) en el dise�o de una base de datos, las entidades se corresponden pr�cticamente con las clases del modelo de dominio, los atributos de dichas entidades se corresponden con los atributos de las clases, y las relaciones se corresponden con las asociaciones que existen entre las diferentes clases del modelo de conocimiento o dominio.

----- Comentar qu� entidades y relaciones se dferencian del modelo de dominio y porqu� -----

El modelo ER de la base de datos se muestra en la imagen \ref{fig:BaseDatosER} y est� accesible en el \textbf{Fichero ....}.

Cabe destacar que este modelo de la base de datos es el mismo para el servidor front-end y para el servidor de respaldo, pues la base de datos de este �ltimo servidor es una copia de seguridad de la base de datos del servidor front-end, por lo que ambas bases de datos deben tener las mismas entidades y relaciones. 

\paragraph{Dise�o l�gico}

Una vez se ha realizado el diagrama ER a partir de las clases y asociaciones del modelo de conocimiento del sistema, se debe traducir dicho diagrama a un modelo relacional o de tablas. Para realizar esto, se han tenido en cuenta las siguientes consideraciones: 

\begin{milista}
	\item Para representar la jerarqu�a de herencia de la entidad \textit{Usuario} y las entidades \textit{Administrador}, \textit{Citador} y \textit{M�dico}, se ha utilizado el patr�n de persistencia \textit{1 �rbol de herencia, 1 tabla}, por lo que s�lo se crear� la tabla \textit{Usuarios}, que agrupar� los atributos de todas las entidades anteriores. Sin embargo, es necesario a�adir un nuevo atributo a la tabla \textit{Usuarios} para indicar el rol del usuario, correspondiente a cada una de las subclases. \\
	La raz�n de utilizar dicho patr�n es agrupar en �nica tabla toda la jerarqu�a de herencia que existe entre esas clases en el modelo de dominio, pues ninguna de las clases \textit{Administrador}, \textit{Citador} ni \textit{M�dico} a�aden nuevos atributos a la clase \textit{Usuario}, por lo que no se van a obtener atributos (columnas) nulas en la tabla resultante.
	\item Para representar los dferentes tipos de m�dico, aunque en el modelo ER se represente con una herencia, por ser un modelo conceptual, la tabla que se obtiene es la tabla \textit{TiposMedico}, obtenida al aplicar el patr�n de persistencia \textit{1 �rbol de herencia, 1 tabla} sobre la clase \textit{Tipo M�dico} que aparece en el modelo de dominio. Esta clase, a su vez, proviene de la aplicaci�n del patr�n \textbf{Estado}, como ya se coment� en el apartado .
\end{milista}



