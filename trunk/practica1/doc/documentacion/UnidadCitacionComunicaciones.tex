\paragraph{Capa de comunicaciones}

Del paquete de comunicaciones del cliente, cabe decir que se utiliza el patr�n
\textbf{Proxy} para comunicarse con el servidor front-end y que se busca de
manera autom�tica una IP p�blica o privada para exportar el objeto remoto del
cliente, de la misma forma que se explic� en la capa de comunicaciones del
servidor frontend (ver secci�n \ref{comunicaciones-front-end}).

Adem�s, al dise�ar la unidad de citaci�n, se ha tenido en cuenta que en una
misma m�quina puede haber varios clientes ejecut�ndose a la vez, algo que no
tiene sentido en los servidores. Por este motivo, el puerto en el que cada
cliente pone su objeto remoto \textit{ICliente} a la escucha no es fijo, sino
que se selecciona din�micamente, probando varios puertos si los que se intentan
utilizar ya est�n en uso en el equipo.
