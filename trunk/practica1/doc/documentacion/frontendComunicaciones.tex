\paragraph{Capa de comunicaciones} \label{comunicaciones-front-end}

En el paquete de comunicaciones se encuentran todas las clases relacionadas
con la conexi�n entre el servidor front-end y la unidad de citaci�n o el
servidor de respaldo. Para facilitar la comprensi�n y el mantenimiento tanto
de la unidad de citaci�n como de los dos servidores, se ha utilizado una
nomenclatura est�ndar para las clases principales de este paquete, de tal
forma que las clases que comienzan por \textit{Remoto...} sirven para exportar
objetos por RMI (como \textit{RemotoServidorFrontend}) y las clases que
comienzan por \textit{Proxy...} se emplean para comunicarse con objetos
remotos de otros sistemas (como \textit{ProxyCliente}).

Como se puede observar, para comunicarse con el cliente y el servidor de
respaldo, se ha hecho uso del patr�n \textbf{Proxy}. Para ello, en este
sistema se han creado dos clases \textit{ProxyCliente} y
\textit{ProxyServidorRespaldo} que son las responsables de establecer conexi�n
con la clases remotas exportadas por cada uno de los clientes registrados en
el sistema y por el servidor de respaldo.

Para comentar c�mo funciona el patr�n \textbf{Proxy} tomaremos como base la
comunicaci�n con el servidor de respaldo, siendo an�loga la comunicaci�n con
los clientes. En este caso, la clase \textit{ProxyServidorRespaldo} implementa
la interfaz \textit{IServidorRespaldo}, la cual contiene todas las operaciones
que se le pueden pedir al servidor de respaldo. De esta forma, el controlador
principal llama a la clase proxy de la misma forma que har�a si el servidor
de respaldo fuera local y estuviera implementado como parte del servidor
front-end pero, siguiendo el patr�n \textbf{Proxy}, la clase
\textit{ProxyServidorRespaldo} reenv�a todas las peticiones al objeto remoto
que est� en la m�quina del servidor de respaldo.

En este apartado conviene destacar que, con el fin de que el sistema funcione
correctamente incluso si la unidad de citaci�n o alguno de los servidores est�
dentro de varias redes, al exportar los objetos remotos se recorren todas las
interfaces de red para buscar una IP seg�n el siguiente orden:
\begin{enumerate}
	\item Si el ordenador pertenece a una red p�blica, se usa una IP p�blica.
	\item Si el ordenador no pertenece a una red p�blica pero s� a una privada,
	se utiliza una IP privada.
	\item Si el ordenador no est� conectado a ninguna red, se emplea la IP
	localhost (127.0.0.1).
\end{enumerate}

Adem�s, para que la comunicaci�n con los objetos remotos se establezca
correctamente, no s�lo es necesario indicar la IP al exportar los objetos, sino
que tambi�n hace falta modificar la propiedad
\texttt{java.rmi.server.hostname} de la m�quina virtual de Java (en los tres
sistemas), que representa la IP del servidor RMI que contiene los objetos.
Esta modificaci�n, sin embargo, se realiza en el controlador principal de cada
sistema, ya que no es responsabilidad de ning�n objeto remoto en particular.
